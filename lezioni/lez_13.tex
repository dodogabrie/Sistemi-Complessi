\section{Processi a salto}%
\label{sub:Lezione 13}
\mylocaltoc \\
Prendiamo di nuovo l'equazione differenziale CK (\ref{eq:4_CK}) e concentriamoci sulla parte che abbiamo trascurato fino a questo punto: il processo a salti.
\subsection{Processo di nascita e morte}%
\label{sub:Processo di nascita e morte}
Prendiamo un processo con la probabilità di transizione $\omega$:
\[
    \omega (x|x',t) = t^+(x') \delta_{x,x'+1} + t^-(x') \delta_{x,x+1}
.\] 
In cui $t^{\pm}(x')$ sono le probabilità di nascita o di morte per un individuo in $x$, quindi per la popolazione $x$ si ha:
\[
    \begin{cases}
	x\to x+1  \text{ con prob. } t^+(x) \\
	x\to x-1 \text{ con prob. } t^-(x) 
    \end{cases}
.\] 
Stiamo valutando come variabile $x$ l'andamento del numero degli individui della popolazione.
La probabilità di trovarsi in un certo punto $x$ è quindi data dalla parte discontinua della equazione di Chapman-Kolmogorov:
\[\begin{aligned}
    \partial_{t}P\left(x,t|x',t'\right) =& t^+(x-1) P\left(x-1, t|x',t'\right) +\\
					 & + t^-(x+1)P(x+1, t| x',t') +\\
					 & - \left(t^+(x) + t^-(x) \right)P\left(x,t|x',t'\right)
.\end{aligned}\]
Proviamo a risolvere questa equazione in modo analogo a quanto fatto per la Fokker-Plank. Cerchiamo la distribuzione di equilibrio:
\[
    \partial_{t}P_s(x) = 0
.\] 
\[\begin{aligned}
    0 =& t^+(x-1) P_s\left(x-1, t|x',t'\right) +\\
       & + t^-(x+1)P_s(x+1, t| x',t') +\\
       & - \left(t^+(x) + t^-(x) \right)P_s\left(x,t|x',t'\right)
.\end{aligned}\]
Quindi possiamo scrivere una equazione per la conservazione della corrente:
\begin{equation}
    J(x+1) - J(x) = 0 \label{eq:13_curr}
\end{equation}
In cui la corrente è definita come:
\begin{equation}
    J(x) = t^-(x) P_s(x) - t^+(x-1) P_s(x-1) \label{eq:13_def_curr}
\end{equation}
Scegliamo le condizioni al contorno: identifichiamo con $x=0$ il primo sito, le condizioni al contorno scelte sono tali per cui la transizione $(x=0)\to (x=-1) $ è posta nulla. In questo modo si modella un processo di nascita e morte in cui il numero di individui di una specie non può scendere sotto lo zero.
\[
    t^-(0) = 0
.\] 
La condizione sul propagatore diventa:
\[
    P\left(x,t|x',t'\right) = 0 \quad \text{ se } x < 0 \ || \ x' < 0
.\] 
Si hanno quindi delle conseguenze su $J(0)$:
\begin{equation}
    J(0) = t^-(0) P_s(0) - t^{-1}P_s(-1) = 0 \label{eq:13_staz}
\end{equation}
Sfruttando l'identità \ref{eq:13_curr} applicata a $(x+1,x),(x,x-1),\ldots$ fino a $(1,0)$ si arriva a dimostrare che, per via della \ref{eq:13_staz}:
\[
    J(x) - J(0) = 0 \implies  J(x) = 0
.\] 
Questo risultato (oltre a dirci che il sistema presenta il bilancio dettagliato) ci permette di trovare la distribuzione di equilibrio utilizzando la definizione di $J(x)$ (\ref{eq:13_def_curr}): 
\[
    P_s(x) = \frac{t^+(x-1)}{t^-(x)}P_s(x-1) 
.\] 
Risolvendo tale equazione definita per ricorrenza:
\[
    P_s(x) = P_s(0) \prod_{z=1}^{x} \frac{t^+(z-1)}{t^-(z)} 
.\] 
\subsection{Applicazione a processo con Rate di passaggio}%
\label{sub:Applicazione a processo con Rate di passaggio}
Immaginiamo un processo in cui i rate di passaggio $t^{\pm}$ possono essere scritti come:
\[
    \begin{cases}
	t^+(x) = k_2a\\
	t^-(x) = k_1 x
    \end{cases}
.\] 
Stiamo studiando un sistema di popolazioni in cui ho l'equilibrio tra la popolazione $X$ e la popolazione $A$ che ha un rate di nascita fissato $k_2a$. \\
La master equation si esprime come:
\[\begin{aligned}
    \partial_{t}P(x,t) = & k_2aP(x-1, t) +\\
			 &+k_1(x+1) P(x+1,t) + \\
			 & - (k_1x+k_2a) P(x,t) 
.\end{aligned}\]
Per trovare la distribuzione di equilibrio utilizziamo la funzione generatrice:
\[
    G(s,t) = \sum_{x=0}^{\infty} s^{x}P(x,t) 
.\] 
Notiamo che vale la seguente:
\[
    xs^xP(x,t) = s\partial_{s}s^xP(x,t) 
.\] 
Quindi otteniamo una equazione per $G(s,t)$:
\[\begin{aligned}
    \partial_{t} G(s,t) =& k_2asG(s,t) + k_1\partial_{s}G(s,t) + \\
			 &- k_1s\partial_{s}G(s,t) - k_2aG(s,t) = \\
                        =& k_2a(s-1) G(s,t) + k_1(1-s) \partial_{s}G(s,t) 
.\end{aligned}\]
Questa equazione alle derivate parziali può essere risolta con il metodo delle caratteristiche. Prima di applicare il metodo è necessario effettuare il cambio di variabile:
\[
    \phi = \ln G
.\] 
\[
    \implies  \phi_t +k_1(s-1) \phi_s - k_2a(s-1) =0 
.\] 
A questo punto si ottengono le due equazioni differenziali che ci portano alla soluzione:
\[\begin{aligned}
    &dt = \frac{ds}{k_1(s-1)}\\
    & \frac{ds}{k_1} = \frac{d\phi}{k_2a}
.\end{aligned}\]
Dalla prima si ottiene:
\[
    u_1=t-\frac{\ln (s-1) }{k_1}
.\] 
dalla seconda invece:
\[
    u_2 = \frac{s}{k_1} - \frac{\phi}{k_2a} = f(e^{-k_1t}(s-1) ) = f(e^{-k_1u_1})
.\] 
Se ne conclude che 
\[
    \phi  = f(e^{-k_1u_1}(s-1)) + \frac{sk_2a}{k_1}
.\] 
E quindi abbiamo la $G$:
\[
    G = \exp\left(\frac{k_2as}{k_1}\right)f(e^{-k_1u_1}(s-1) ) 
.\] 
Notiamo che all'interno dell'esponenziale abbiamo $s$ mentre nella funzione $f$ compare il termine $(s-1)$. Possiamo allora moltiplicare la $G$ per una costante (essendo $f$ arbitraria) e riscrivere il tutto come:
\begin{equation}
    G = \exp\left(\frac{k_2a(s-1) }{k_1}\right)f(e^{-k_1u_1}(s-1) ) \label{eq:13_G}
\end{equation}
Non sarebbe possibile sostituire $s-1 \to s$ all'interno di $f$ perché così facendo si sbaglierebbe la dipendenza dal tempo della funzione $u_1$ ($t$ e $\ln (s-1)$ hanno lo stesso andamento poiché $u_1$ costante).\\
Prima di procedere possiamo notare che, per la proprietà di completezza di $P(x,t)$ si ha:
\[
    G(1,t) = \sum_{}^{} P(x,t) = 1
.\] 
Questo implica nella \ref{eq:13_G} che:
\[
    f(0) = 1
.\] 
Nel metodo delle caratteristiche per trovare la forma della $f$ dobbiamo inserire le condizioni iniziali. 
\subsubsection{Condizioni iniziali a $\delta$.}%
\label{subsub:Condizioni iniziali delta}
supponiamo che nello stato iniziale la popolazione sia $N$ (vuol dire che la probabilità diventa una $\delta$ in $N$):
\[
    P(x,0|N,0) = \delta_{x,N}  \ \implies  \  G(s,0) = s^N
.\] 
In cui $N$ è un numero arbitrario. Sostituendo nella \ref{eq:13_G} con $t=0$ si ottiene:
\[
    f(s-1) \exp\left(\frac{k_2a}{k_1}(s-1)\right) = s^N
.\] 
Invertendo l'espressione si ha la $f$:
\[\begin{aligned}
    f(s-1) =& s^N\exp\left((s-1) \frac{k_2a}{k_1}\right) =\\
    =& \exp\left(-(s-1) \frac{k_2a}{k_1}\right)(s-1 + 1) ^N
.\end{aligned}\]
Adesso reintroduciamo questa sempre nella \ref{eq:13_G} ricordando di sostituire, nella $f$ il termine $s-1$ con $(s-1) e^{-k_1t}$.
\[\begin{aligned}
    G(s,t) = \exp\left(\frac{k_2a}{k_1}(s-1)\right)& ((s-1) e^{-k_1t} + 1)^N \cdot  \\ 
	    &\cdot \exp\left(-(s-1)e^{-k_1t} \frac{k_2a}{k_1}\right)
.\end{aligned}\]
Il problema è formalmente risolto, tuttavia non si risale analiticamente al propagatore $P$ tramite questa funzione caratteristica. \\
La funzione $G$  ci permette comunque di trovare il valor medio della popolazione:
\[\begin{aligned}
    &\left<x\right>_t = \left.\partial_{s}G(s,t)\right|_{s=1} =\frac{k_2a}{k_1}(1-e^{-k_1t}) + Ne^{-k_1t}\\
    &\left<x^2\right>_t = \left.\partial^2_{s^2}G(s,t)\right|_{s=1} = \left<x\right>_t^2 - Ne^{-2k_1t}
.\end{aligned}\]
Visto che asintoticamente il sistema presenta una caratteristica tipica della Distribuzione di Poisson: $\sigma^2\to \mu$. Inoltre il valore della media per $t\to \infty$ è quello atteso se si scrive una semplice equazione del moto per la $x$ a stazionarietà:
\[
0 = \dot{x} = -k_1x + k_2a \quad \implies  \quad x_s = \frac{k_2a}{k_1}
.\] 
\subsubsection{Condizioni iniziali Poissoniane}%
\label{subsub:Condizioni iniziali Poissoniane}
Supponiamo che la popolazione iniziale avesse una distribuzione di probabilità poissoniana del tipo:
\[
    P(x,0) = e^{-a_0}\frac{a_0^x}{x!}
.\] 
In tal caso la funzione caratteristica all'istante iniziale si può riscrivere tramite lo sviluppo dell'esponenziale:
\[\begin{aligned}
    G(s,0) =& \sum_{}^{} s^xP(x,0) =\\
	    & = \sum_{x}^{} (sa_0) ^x \frac{e^{-a_0}}{x!} =\\
	    & = e^{-a_0}e^{sa_0} = e^{a_0(s-1) }
.\end{aligned}\]
Con passaggi analoghi a quelli per il caso precedente siamo in grado di trovare la $f$  e reinserirla in $G(s,t)$  ottenendo:
\[\begin{aligned}
    G(s,t) =& \exp\left((s-1) \frac{k_2a}{k_1}\right)\times  \\
	    &\times  \exp\left((s-1)e^{-k_1t}\left(a_0- \frac{k_2a}{k_1}\right)\right)=\\
           =& \exp\left((s-1) \left[a_0 e^{-k_1t} + \frac{k_2a}{k_1}(1-e^{-k_1t})\right]\right)
.\end{aligned}\]
Notiamo che la struttura della trasformata di una poissoniana è rimasta anche nella $G(s,t)$, solo che a differenza della $G(s,0)$  il termine $a_0$  è stato sostituito da un termine $\alpha (t) $  tale che:
\[
    \alpha (t) = a_0 e^{-k_1t} + \frac{k_2a}{k_1}(1-e^{-k_1t})
.\] 
Quindi:
\[
    G(s,t) = \exp\left((s-1) \alpha (t) \right)
.\] 
Applicando un'antitrasformata possiamo risolvere nello spazio reale:
\[
    P(x,t) = \frac{e^{-\alpha (t)}}{x!}\left[\alpha (t) \right]^x
.\] 
Quindi la soluzione con queste condizioni iniziali è una poissoniana "traslata".\\
Si ha che in generale che lavorare con la Master Equation è complicato ma per questo caso specifico siamo stati in grado di ottenere la distribuzione.\\
Potrebbe essere utile trovare il sistema per passare dalla Master-Equation alla Fokker-Plank in modo da applicare i metodi risolutivi discussi nelle precedenti sezioni.
\clearpage
