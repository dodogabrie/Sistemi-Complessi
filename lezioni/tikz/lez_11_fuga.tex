\tikzmath{\a = 1; \b = 3;}
\usetikzlibrary{patterns}
\usepgfplotslibrary{fillbetween}

\begin{figure}[H]
    \centering
    \begin{tikzpicture}
	\begin{axis}[
	    width=7cm,
	    height=4cm,
	    xmin= -1, xmax= 5,
	    ymin= 0, ymax = 2.,
	    axis lines = middle,
	    x label style={at={(axis description cs:1,-0.01)},anchor=north},
	    y label style={at={(axis description cs:0.15,1)},anchor=south},
	    xlabel={$x$},
	    ylabel={$U(x)$ },
	    xtick={ 0, \a+0.1, 1.9},
	    xticklabels={$0$, $a$, $b$},
	    ytick={0},
	    yticklabel={$0$},
	    ]
	    \addplot[domain=0.2:5, samples=500]{((x-\a)^2+0.2)*(x-\b)^2 + 0.3};
	    \addplot[densely dotted, samples=50, smooth,domain=0:6, name path=one] coordinates {(\a+0.1,0)(\a+0.1,1.)};
	    \addplot[densely dotted, samples=50, smooth,domain=0:6, name path=two] coordinates {(1.9,0)(1.9,1.5)};
	\end{axis}
    \end{tikzpicture}
    \caption{\scriptsize Potenziale al quale sono soggetti i camminatori.}
    \label{fig:lore}
\end{figure}
