\begin{figure}[H]
    \centering
    \begin{tikzpicture}
  	\tikzmath{\xa = 2; \y = 3; \lim = 1.6;}
	\begin{axis}[
	    width=7cm,
	    height=4cm,
	    xmin= -\xa, xmax= \xa,
	    ymin= 0, ymax = \y,
	    axis lines = middle,
	    x label style={at={(axis description cs:1.05,0.06)},anchor=north},
	    y label style={at={(axis description cs:0.5,1)},anchor=south},
	    xlabel={$x$},
	    ylabel={$U(x)$ },
	    xtick={0},
	    xticklabels={$0$},
	    ytick={0},
	    yticklabel={$0$},
	    ]
	    \addplot[domain=-3:3, samples=500]{((x+1)^2+0.3)*(x-1)^2};
	\end{axis}
    \end{tikzpicture}
    \begin{tikzpicture}
  	\tikzmath{\xa = 2; \y = 3; \lim = 1.6;}
	\begin{axis}[
	    width=7cm,
	    height=5cm,
	    xmin= -\xa, xmax= \xa,
	    ymin= -\y, ymax = \y,
	    axis lines = middle,
	    x label style={at={(axis description cs:1.05,0.56)},anchor=north},
	    y label style={at={(axis description cs:0.5,1)},anchor=south},
	    xlabel={$x$},
	    ylabel={$v$},
	    xtick={0},
	    xticklabels={$0$},
	    ytick={0},
	    yticklabel={},
	    ]
	    \foreach \s in {0,1,2, 2.5} { % Traccio più linee per avere più punte di freccia
		\foreach \E in {1.354 , 1.6, 2} { % Vario l'energia per vedere le tre aree dello spazio delle fasi
		    %%%%%%%%%%%%%%%%%%%%%%%%%%%%%%%%%%%%%%%%%%%%%
		    %  Linee del moto al variare della energia  %
		    %%%%%%%%%%%%%%%%%%%%%%%%%%%%%%%%%%%%%%%%%%%%%
		    %((x+1)^2+0.5)*(x-1)^2
		    %(-x^2/2*(1-x^2/2) + 0.3)
		    \addplot [domain= ( -\lim): ( \lim) - 2*\lim*\s/3, samples=300, ->] {sqrt( 2*(\E - ((x+1)^2+0.3)*(x-1)^2)  )};
		    \addplot [domain= ( -\lim) + 2*\lim*\s/3: ( \lim) , samples=300, <-] {-sqrt( 2*(\E - ((x+1)^2+0.3)*(x-1)^2) )};
		}
	    }
	    \addplot [domain= ( 0): ( 1.9), samples=200, ->] {sqrt( 2*(0.4 - ((x+1)^2+0.3)*(x-1)^2)  )};
	    \addplot [domain= ( 0): ( 1.9) , samples=200, <-] {-sqrt( 2*(0.4 - ((x+1)^2+0.3)*(x-1)^2) )};
	    \addplot [domain= ( -1.9): ( 0), samples=200, ->] {sqrt( 2*(1.2 - ((x+1)^2+0.3)*(x-1)^2)  )};
	    \addplot [domain= ( -1.9): ( 0) , samples=200, <-] {-sqrt( 2*(1.2 - ((x+1)^2+0.3)*(x-1)^2) )};
	\end{axis}
    \end{tikzpicture}
    \caption{\scriptsize Potenziale a doppia buca e linee di flusso per il moto nello spazio delle fasi. Notiamo la presenza di un punto sella che equivale ad un moto nel quale l'oggetto raggiunge il punto di massimo tra le buche con velocità nulla (in un tempo infinito). }
    \label{fig:15_double}
\end{figure}
