\section{Lezione 8}%
\label{sub:Lezione 8}
\subsection{Legame tra SDE e Fokker-Plank}%
\label{sub:Legame tra SDE e Fokker-Plank}
Prendiamo una equazione differenziale sticastica del tipo:
\[
    dx = adt + bd\omega
.\] 
Possiamo immaginare che questa SDE dia luogo ad una distribuzione di probabilità Markoviana, quindi che soddisfi l'equazione di Chapman-Kolmogorov (\ref{eq:4_CK}). \\
Il problema è che la forma differenziale di CK è molto generale, cerchiamo di capire quale forma assume per soddisfare la SDE sopra.\\
Prendiamo una generica funzione $f(x(t))$, il suo differenziale è dato dalla formula di $\hat{I}$to:
\[
    df = \left(a\partial_{x}f + \frac{1}{2}b^2\partial^2_{x^2}f\right)dt + b \partial_{x}f d\omega
.\] 
Consideriamo la derivata di $f$ rispetto al tempo mediata sulle realizzazioni di $\omega$:
\[
    \left<\frac{\text{d} f}{\text{d} t} \right>_{\omega} = \frac{\text{d} }{\text{d} t} \left<f(x(t) )\right>
.\] 
Essendo $\left<d\omega\right>_{\omega}=0$ si ha che:
\begin{equation}
    \frac{\text{d} \left<f\right>}{\text{d} t} = \left<a \partial_{x}f +\frac{1}{2}b^2\partial^2_{x^2}f\right>
    \label{eq:8_dt}
\end{equation}
Si assume che l'oggetto sia partito da $\left(x_0, t_0\right)$, l'equazione di CK ci permette di esprimere $\left<f(x(t))\right>$ in termini del propagatore $P\left(x,t|x_0, t_0\right)$:
\[
    \left<f(x(t) ) \right> =  \int  dx f(x) P\left(x,t|x_0, t_0\right)
.\] 
In tale espressione la dipendenza temporale entra soltanto all'interno del propagatore. Di conseguenza quando la si deriva rispetto a $t$ la derivata agisce solo su $P \equiv P(x,t|x_0,t_0)$:
\[
   \frac{\text{d} }{\text{d} t} \left<f(x(t) ) \right> =  \int  dx f(x) \frac{\partial }{\partial t}  P\left(x,t|x_0, t_0\right)
.\] 
A sinistra si sostituisce la \ref{eq:8_dt}:

\[
    \int dx \left[a \partial_{x}f +\frac{1}{2}b^2\partial^2_{x^2}f\right]P = \int dx f(x) \partial_{t}P 
.\] 
Integrando per parti a destra dell'uguale e supponendo che la $P\left(x,t|x_0,t_0\right)$ non diverga al bordo:
\[
    \int dx f(x) \partial_{t}P = 
    \int dx f(x) \left[-\partial_{x}(aP) + \partial^2_{x^2}\left(\frac{1}{2}b^2P\right)\right]
.\] 
Visto che la funzione $f(x)$ è arbitraria si ottiene ha la forma di una CK come anticipato:
\begin{redbox}{Legame tra SDE e C-K}
    La forma differenziale stocastica:
    \[
    dx = adt + bd\omega
    .\] 
    Conduce alla equazione di Fokker-Plank (o forma differenziale di CK):
    \begin{equation}
    \partial_{t}P(x,t) = \left(-\partial_{x}a + \frac{1}{2}\partial^2_{x^2}b^2\right)P(x,t) 
    \label{eq:8_CK-SDE}
    \end{equation}
\end{redbox}
\noindent
\begin{exmp}[]
    Prendiamo i seguenti valori per i parametri della SDE:
    \begin{itemize}
        \item $a(x,t) = a(t)$
	\item $b(x,t) = b(t)$
    \end{itemize}
    \[
	dx = a(t)dt + b(t) d\omega
    .\] 
    Integrando si ha: 
    \[
	x(t) = x_0 + \int_{0}^{t} a(s) ds + \int_{0}^{t} b(s) d\omega_s 
    .\] 
    Mediando sulle realizzazioni di $\omega$ l'ultimo termine va via:
    \[
	\left<x(t) \right>_\omega = \left<x_0\right>+\int_{0}^{t} a(s) ds 
    .\] 
    Calcoliamo anche la varianza:
    \[\begin{aligned}
	\left<x(t) x(s) \right> =& \left<\left(x(t) - \left<x(t)\right>\right)\left(x(s) - \left<x(s)\right>\right)\right> = \\
				 & = \left<\int_{0}^{t} b(t') d\omega(t') \int_{0}^{s} b(s') d\omega (s')\right>
    .\end{aligned}\]
    Sfruttando le proprietà della varianza per un processo di Wiener:
    \[
	\left<x(t) x(s) \right> = \int_{}^{\text{min}(t,s)} b^2(t') dt'
    .\] 
    Nel caso più semplice in cui $a, b$ costanti:
    \begin{itemize}
	\item $\left<x(t)\right>=x_0+at$ 
	\item $\left<x(t) x(s)\right> = b^2 \text{min}(t,s)$ 
    \end{itemize}
    Che sono gli stessi risultati ottenuti nel caso del moto Browniano.
\end{exmp}
\noindent
\begin{exmp}[]
    \[
	dx = cxd\omega (t) 
    .\] 
    Potremmo procedere con l'approccio di Stratonovich:
    \[
	\frac{dx}{x} = dy = cd\omega (t) 
    .\] 
    Il problema è che non è detto che l'oggetto a sinistra sia morbido (che sia un processo che si può risolvere con l'integrale di Stratonovich), quindi questo approccio in generale potrebbe non essere corretto.\\
    Nel caso in cui il processo segua una dinamica "alla $\hat{\text{I}}$to" (ad esempio un processo a salti) è necessario utilizzare la formula di $\hat{\text{I}}$to per effettuare il cambio di variabili. Prendiamo il seguente:
    \[
        f = y = \ln x
    .\] 
    La formula ci dice che:
    \[
        df = \left(af'+ \frac{1}{2}b^2f''\right)dt + bf'd\omega
    .\] 
    Nel nostro caso: 
    \begin{itemize}
        \item $a=0$ 
	\item $b = cx$
	\item $f' = 1 /x$ 
	\item $f'' = - 1/x^2$ 
    \end{itemize}
    Quindi in conclusione si ha una equazione differenziale per $y$ che non è quella che ci saremmo aspettati:
    \[
        dy = -\frac{c^2}{2}dt + cd\omega
    .\] 
    Abbiamo in più il primo termine. Integrando:
    \[
	y(t) = y_0 + c\omega(t) - \frac{c^2}{2}t
    .\] 
    A questo punto il problema è risolto per $x$:
    \[
	x(t) = \exp\left(y\right) = x_0\exp\left(c\omega (t) - \frac{c^2}{2}t\right)
    .\] 
    In cui si evidenzia ancora una volta il fatto che $\omega (t)$ è un processo di Wiener, quindi è interessante capire cosa avviene quando effettuiamo la media su diversi processi (quindi su diverse realizzazioni di $\omega )$.\\
    Possiamo calcolare $\left<x\right>_\omega$ sfruttando il fatto che il valor medio di un processo gaussiano è nullo.
    \[
	z \in G(0, 1) \implies  \left<z\right> = 0
    .\] 
    Nella nostra equazione abbiamo una espressione del tipo $\left<\exp (z)\right>$, sfruttando le proprietà dei momenti di un processo Gaussiano si ha che:
    \begin{equation}
	\left<\exp\left(z\right)\right> = \exp\left(\frac{\left<z^2\right>}{2}\right) \label{eq:8_gauss}
    \end{equation}
    Per dimostrarlo è necessario utilizzare lo sviluppo dell'esponenziale, i momenti maggiori del secondo si annullano e rimane soltanto quello.\\
    Otteniamo in conclusione che:
    \[\begin{aligned}
	\left<x(t)\right>=& \left<x_0\right>\exp\left(-\frac{c^2}{2}t\right)\left<\exp\left(c\omega (t) \right)\right> = \\
			  & =\left<x_0\right>\exp\left(-\frac{c^2}{2}t\right) \exp\left(\frac{c^2}{2}t\right) = \left<x_0\right>
    .\end{aligned}\]
    Analogamente si può fare con la correlazione:
    \[\begin{aligned}
	\left<x(t) x(s)\right> = & \left<x_0^2\right>e^{-\frac{c^2}{2}\left(t+s\right)} \left<e^{c(\omega (t) + \omega (s) )}\right> = \\
				 & =\left<x_0^2\right>e^{-\frac{c^2}{2}\left(t+s\right)} e^{\left<\frac{c^2}{2}(\omega(t) +\omega(s))^2\right>} =\\
				 & = \left<x_0^2\right>e^{c^2\text{min}(t,s)}
    .\end{aligned}\]
    In cui si è sfruttata la seguente:
    \[\begin{aligned}
	\left<(\omega(t) \left.&+\right.\omega(s))^2\right> = \\
	&= \left<\omega (t)^2\right> + \left<\omega(s) ^2\right> + 2 \left<\omega (t) \omega (s)\right> = \\
	&= t + s + 2\mbox{min}(t,s) 
    .\end{aligned}\]
    Se avessimo fatto il conto con Stratonovich avremmo ottenuto delle quantità divergenti:
    \[\begin{aligned}
	&\left<x(t) \right> = \left<x_0\right>\exp\left(\frac{1}{2}c^2t\right)\\
	&\left<x_tx_s\right> = \left<x_0^2\right>\exp\left(\frac{1}{2}c^2\left(t+s+2\text{min}(t,s) \right)\right)
    .\end{aligned}\]
    Quindi i due metodi di integrazione portano a dinamiche completamente differenti, è necessario stare attenti ad usare di volta in volta il metodo più opportuno.
\end{exmp}

\begin{exmp}[Oscillatore Kubo]
Si studia la precessione di uno spin attorno ad un campo magnetico $\omega$:
\[
    dz = i\left(\omega dt + \sqrt{2\gamma} d\omega_t\right)z
.\] 
Il secondo termine indica che il campo magnetico non è costante, contiene fluttuazioni $d\omega$. Come conseguenza vedremo che il pacchetto di spin inizierà a sparpagliarsi.\\
Visto che le fluttuazioni del campo devono avere un Cut-Off ad alte frequenze è opportuno usare l'integrazione "fisica" di Stratonovich.\\
Possiamo valutare il valor medio di $z$ integrando nel modo a noi noto:
\[
    \frac{dz}{z}=i\omega t + i\sqrt{2\gamma} d\omega_t
.\] 
La soluzione per $z$ è ovviamente l'esponenziale del termine di destra, facendo il valor medio e sfruttando la \ref{eq:8_gauss} si ottiene:
\[
    \frac{\text{d} }{\text{d} t} \left<z\right> = \left(i\omega-\gamma\right)\left<z\right>
.\] 
Come accennato il primo termine fa girare lo spin, il secondo lo sparpaglia.
\[
    \left<z_t\right> = \left<z_0\right>\exp\left(\left(i\omega-\gamma\right)t\right)
.\] 
Essendo in questo caso $z$ una quantità complessa possiamo calcolare una correlazione del tipo:
\[
    \left<z_tz^*_s\right> =\ldots= \left<z_0^2\right>e^{i\omega\left(t-s\right)-\gamma\left|t-s\right|}
.\] 
La funzione di correlazione decade esponenzialmente con un tempo $1 /\gamma$, legato alla fluttuazione del campo magnetico.
\end{exmp}
\noindent
\begin{exmp}[]
    \[
        dx = -kx dt + \sqrt{D} d\omega_t
    .\] 
    Questa è "parente" del processo di Ornstein-Uhlenback:
    \[
	dx = f(x) dt + \sqrt{D} d\omega_t 
    .\] 
    Per risolverla si parte dalla omogenea:
    \[
        dx = fdt = -kxdt
    .\] 
    Visto che il termine di rumore è costante:
    \[
        g = \sqrt{D} \implies  
	\begin{cases}
	    g = \text{cost}\\
	    g' = 0
	\end{cases}
    .\] 
    Allora in questo caso $\hat{\text{I}}$to e Stratonovich conducono allo stesso risultato.\\
    Utilizziamo il calcolo di $\hat{\text{I}}$to, la prima cosa da fare è cercare il giusto cambio di variabile. Scegliamo:
    \[
        y = x e^{kt}
    .\] 
    La formula di $\hat{\text{I}}$to per funzioni dipendenti dal tempo si scrive come:
    \[
        df = \left[a\partial_{x}f + \frac{b^2}{2}\partial^2_{x^2}f + \partial_{t}f\right]dt + b\partial_{x}fd\omega
    .\] 
    Sviluppando le derivate si ottiene che:
    \[
        dy = \sqrt{D} e^{kt}d\omega
    .\] 
    E quindi tornando indietro abbiamo anche la $x$:
    \[
	x(t)  = x_0 e^{-kt}+\sqrt{D} \int_{0}^{t} e^{-k(t-t')}d\omega_{t'} 
    .\] 
    Mediando nel tempo nuovamente i termini con $d\omega$ si cancellano:
    \[
	\left<x(t) \right> = \left<x_0\right>e^{-kt}
    .\] 
    Per la varianza il calcolo è più elaborato, riportiamo la conclusione:
    \[\begin{aligned}
	&\text{var}\left\{x(t) \right\} =\\ 
	& =\left<\left[(x_0-\left<x_0\right>)e^{-kt} + \sqrt{D} \int_{}^{t} e^{-k(t-t') }d\omega_{t'}\right]^2 \right> =\\
	& =e^{-2kt}\left[\text{var}\left\{x_0\right\}-\frac{D}{2k}\right]+ \frac{D}{2k}
    .\end{aligned}\]
    Quindi la varianza ha un valore stazionario ed un termine che decade esponenzialmente.
\end{exmp}
\noindent
\subsection{Ornstein-Uhlenback dipendente dal tempo}%
\label{sub:Ornstein-Uhlenback dipendente dal tempo}
Prendiamo la seguente SDE:
\[
    dx = -a(t) x dt + b(t) d\omega
.\] 
L'algebra da seguire è simile a quella dell'esempio precedente, risolviamo l'omogenea (senza $\omega$):
\[
    x(t) = \exp\left(-\int_{0}^{t} a(s) ds \right)x_0
.\] 
Si inserisce adesso la parte disomogenea:
\[\begin{aligned}
    x(t) =& x_0 \exp\left(-\int_{0}^{t} a_sds \right) + \\
	  & + \int_{0}^{t} \exp\left(-\int_{t'}^{t} a(s) ds  \right)b(t') d\omega_{t'} 
.\end{aligned}\]
Al solito si può mediare in $\omega$ per mandare via il secondo integrale:
\[
    \left<x(t)\right> = \left<x_0\right> \exp\left(-\int_{0}^{t} a(s) ds \right)
.\] 
Mentre per la covarianza si ha che:
\[\begin{aligned}
    \left<x(t),x(t) \right> = & \exp\left(-2  \int_{0}^{t} a(s) ds \right)\left<x_0,x_0\right> + \\
			      & + \int_{0}^{t} dt'  \exp\left(-2\int_{t'}^{t} a(s) ds \right) b^2(t') 
.\end{aligned}\]
\clearpage
