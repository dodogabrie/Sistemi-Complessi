\section{Metodi analitici per la soluzione di FK}%
\label{sub:Lezione 11-1}
\mylocaltoc
\subsection{Equazione FP in forma conservativa.}%
\label{sub:Equazione di Fokker-Plank: soluzione analitica.}
Abbiamo visto nella Sezione \ref{sub:Lezione 4} la forma differenziale di Chapman-Kolmogorov (equazione \ref{eq:4_CK}). La parte di tale equazione che conteneva termini di derivate prime e seconde (la parte continua del processo) è nota come:
\begin{redbox}{Equazione di Fokker-Plank.}
   \begin{align}
       \partial_{t}P(\vect{x}, t) = &
      				      -\partial_{x_i}A_i(\vect{x}, t) P(\vect{x}, t) + \nonumber \\
				    & + \frac{1}{2}\partial_{x_i}\partial_{x_J}B_{iJ}(\vect{x},t) P(\vect{x},t) 
				    \label{eq:FK_general}
   .\end{align}
\end{redbox}
\noindent
Il termine $B_{iJ}$ rappresenta la diffusione del processo.
\paragraph{Corrente di probabilità e forma conservativa}
Definiamo una corrente $\vect{J}$:
\[
    J_i = \left(A_i - \frac{1}{2}\partial_{x_J}B_{iJ}\right)P(\vect{x},t) 
.\] 
Quindi l'equazione di FK si scrive come:
\[
    \partial_{t}P + \nabla \vect{J} = 0
.\] 
Quindi possiamo integrare $P$ su un volume $V$ e sfruttare il teorema di Gauss:
\[\begin{aligned}
    \partial_{t}\int\limits_{V} PdV = - \int\limits_{V}  \nabla \vect{J} dV = - \int\limits_{\partial V} d\vect{s}\cdot \vect{J}
.\end{aligned}\]
Dove $\partial V$ è il bordo del volume $V$ (e l'ultima espressione dopo l'uguale è il flusso di $\vect{J}$ sul bordo di $V$). 
\begin{greenbox}{}
    La variazione della probabilità in un volume $V$ è legata al flusso della corrente di probabilità.
\end{greenbox}
\noindent
Quindi la soluzione di questa equazione alle derivate parziali dipende fortemente dall'andamento della corrente lungo la superficie di $V$.
\subsection{Condizioni al bordo per la FK.}%
\label{sub:Condizioni al bordo per la soluzione della FK.}
\paragraph{Condizioni al bordo riflettenti.}%
Questo è il caso in cui la corrente di probabilità rispetta la seguente:
\[
    \hat{S}\vect{J} (\vect{z}, t) = 0
.\] 
Con $\hat{S}$ versore del bordo di $V$. Questa condizione implica che la $\vect{J}$ è parallela alla superficie.
\paragraph{Condizioni al bordo assorbenti.}%
\[
    \text{Se }\vect{z}  \in \partial S \implies  J(\vect{z}, t) = 0
.\] 
\paragraph{Condizioni al bordo discontinue}%
La corrente può essere discontinua sul bordo, in queste condizioni il moto resta possibile, la corrente deve soddisfare l'equazione:
\[
    \hat{S}\left(\vect{J} (\vect{z}^+) - \vect{J} (\vect{z}^-) \right) = 0
.\] 
\paragraph{Condizioni al bordo periodiche.}%
Mettiamoci in una dimensione ad esempio, nel segmento $\left[a, b\right]$, allora le condizioni al contorno periodiche implicano che:
\[\begin{aligned}
    & P(a^+) = P(b^-) \\
    & \vect{J} (a^+) = \vect{J} (b^-) 
.\end{aligned}\]
\paragraph{Condizioni al bordo all'infinito.}%
Tipicamente si lavora con oggetti che possono trovarsi a distanze finite, in tal caso le condizioni all'infinito vengono prese come:
\[\begin{aligned}
    & P(\vect{x}, t) \to 0 \quad \left|x\right|\to \infty\\
    & \vect{J} (\vect{x}, t) \to 0 \quad \left|x\right|\to \infty
.\end{aligned}\]
Se così non fosse allora una delle due quantità dovrebbe divergere.
\paragraph{Diffusione nulla al bordo.}%
Se $B_{iJ} = 0$ sul bordo del volume abbiamo una situazione totalmente deterministica. In tal caso tale zona potrebbe essere: 
\begin{itemize}
    \item Un ingresso dei camminatori
    \item Una uscita dei camminatori
\end{itemize}
In questo caso possono esistere delle situazioni ancor più esoteriche in cui oltre che ad avere $B_{iJ} = 0$ sul bordo si ha anche $A(\vect{z}) = 0$. In questo caso speciale i camminatori che raggiungono il bordo si fermano.
\begin{exmp}[]
    \[
        dx = -\alpha xdt + xd\omega
    .\] 
    In questo caso il camminatore che arriva nell'origine si ferma. L'equazione di FK in questo caso può esser scritta riprendendo la formula \ref{eq:8_CK-SDE}:
    \[
        \partial_{t}P = \left[\partial_{x}\alpha x + \frac{1}{2}\partial^2_{x^2} x^2\right] P
    .\] 
\end{exmp}
\noindent
\subsection{Distribuzione di equilibrio}%
\label{sub:Distribuzione di equilibrio del processo}
La distribuzione di equilibrio è data dalla soluzione stazionaria dell'equazione di Fokker-Plank:
\[
    \partial_t P = 0 \ \implies  \ \nabla J = 0
.\] 
Riscriviamo la corrente esplicitando la derivata $\partial_{x_J}$:
\begin{equation}
    J_i = \left[A_i-\frac{1}{2}\left(\partial_{x_J}B_{iJ}\right)\right]P - \frac{1}{2} B_{iJ}\partial_{x_J}P
    \label{eq:11_J}
\end{equation}
Cerchiamo una distribuzione di equilibrio con le seguenti ipotesi:
\begin{itemize}
    \item $\vect{J}$ si conserva e all'equilibrio $J=0$ .
    \item All'infinito la distribuzione di probabilità si annulla.
    \item $B_{iJ}$ invertibile.
\end{itemize}
Si effettuano alcuni passaggi algebrici nella \ref{eq:11_J} per isolare la $P$:
\begin{align}
    &\partial_{x_J}P \equiv \nabla_J P = 2B^{-1}_{iJ}\left[A_i-\frac{1}{2}\partial_{x_k}B_{ik}\right]P \label{eq:P_staz}\\
    &\implies
    \nabla_{J} (\ln (P)) = 2B_{iJ}^{-1}\left[A_i-\partial_{x_k}B_{ik}\right] \equiv Z_J[A, B, \vect{x}]
    \nonumber
.\end{align}
In cui si è definito il vettore $\vect{Z}$ le cui componenti sono così definite:
\[
    Z_J\left[A,B, \vect{x}\right] \equiv 2B_{iJ}^{-1}\left[A_i - \partial_{x_k}B_{ik}\right]
.\] 
Si applica il rotore a destra e sinistra ricordando che:
\[
    \text{rot}(\nabla f)  = 0 \quad \forall f \implies 
    0 = \mbox{rot}\left[\nabla (\ln (P))\right]  = \text{rot}(\vect{Z})
\]
Il rotore nullo di $\vect{Z}$ impone delle condizioni sulle derivate di tale quantità:
\begin{equation}
    \text{rot}(\vect{Z}) = 0 \implies  
     \frac{\partial }{\partial x_i} Z_J = \frac{\partial }{\partial x_J} Z_i  
    \label{eq:11_rot}
\end{equation}
Se vale questa condizione possiamo allora integrare l'equazione per $\nabla (\ln (P))$ lungo una curva generica:
\[\begin{aligned}
    &\nabla (\ln (P_{st})) = \vect{Z} \\
    & \implies  P_{st}(\vect{x}) = \exp\left(\int\limits^{x} d\vect{s}\vect{Z}\left[A,B, \vect{s}\right]\right)
.\end{aligned}\]
\begin{redbox}{Soluzione stazionaria: forma potenziale}
    \begin{equation}
	P_{st}(\vect{x})  = \exp (-\phi (\vect{x})) 
	\label{eq:11_forma_pot}
    \end{equation}
     \[
	\phi (\vect{x}) \equiv - \int\limits^{x} d\vect{s}\vect{Z} 
    .\] 
\end{redbox}
\noindent
Il metodo descritto in questa sezione non è generale poiché non è sempre possibile esprimere $\text{rot}(\vect{Z}) = 0$.
\begin{exmp}[Processo di Wiener]
    \[
	dx = f(x) dt + Bd\omega
    .\] 
    I passaggi per arrivare alla soluzione stazionaria sono semplici:
    \[\begin{aligned}
	&\partial_{t}P = \left(- \partial_{x}f + \frac{1}{2} B^2\partial^2_{x^2}\right)P = \partial_x \left[\left(-f + \frac{1}{2}B^2\partial_x\right)P\right] \\
	&\implies J = 0 = - f P + \frac{1}{2}B^2\partial_x P \implies  \partial_x (\ln (P) ) = \frac{2f}{B^2}
    .\end{aligned}\]

    \[
	Z = \frac{2f}{B^2} \implies
	P_{st}\sim \exp\left(\int\limits^{x} \frac{2f}{B^2}dx \right) 
    .\] 
    \begin{greenbox}{}
        I processi di Wiener hanno una distribuzione di equilibrio.
    \end{greenbox}
    \noindent
    Il comportamento asintotico della $f$ decreta la bontà della soluzione stazionaria (prendere $f = x$ non va bene).
\end{exmp}
\noindent

\begin{exmp}[Processo di OU]
    Prendiamo la stessa equazione dell'esempio precedente
    \[
	dx = f(x) dt + Bd\omega
    .\] 
    tuttavia adesso il termine $Bd\omega$ anziché essere un processo di Wiener (come si sottointende di solito) lo ipotizziamo di Ornstein-Ulhenback:
    \[
	\begin{cases}
	    Bd\omega  = Bydt\\
	    dy = -\dfrac{1}{\tau}ydt + d\omega
	\end{cases}
    \] 
    Dove il $d\omega$ nel sistema è adesso un processo di Wiener.\\
    Ricordiamo che questo processo è caratterizzato da avere un taglio in frequenza di $\frac{1}{\tau}$.\\
    Per semplificare ancora i conti inseriamo all'interno della SDE un secondo processo di Wiener $d\omega_2$ (si capirà meglio sotto), si ottiene allora il sistema:
    \[
        \begin{cases}
	    dx = \left(f(x) + By\right)dt + d\omega_2\\
	    dy = -\dfrac{1}{\tau}y dt + d\omega_1
        \end{cases}
    .\] 
    In questo caso è evidente che i termini della FK in due dimensioni
    \footnote{Si tratta di due dimensioni perché abbiamo una coppia di variabili stocastiche: $x$ e $y$.}
     corrispondono alle quantità:
    \[\begin{aligned}
	&A_x = f+By & \quad B_{xx} = 1\\
	&A_y = -\frac{1}{\tau}y &\quad B_{yy}=1
    .\end{aligned}\]
    Possiamo scrivere la matrice del rumore $B$ considerando i coefficienti a moltiplicare i due processi di Wiener $d\omega_1, d\omega_2$ nelle equazioni. Visto che i due processi sono indipendenti e compaiono uno per l'equazione di $x$ e l'altro per l'equazione di $y$ si ha:
    \[
        B = 
	\begin{pmatrix}
	    B_{xx} & B_{xy}\\
	    B_{yx} & B_{yy}
	\end{pmatrix}
	=
	\begin{pmatrix}
	    1 & 0\\
	    0 & 1
        \end{pmatrix}
    .\] 
    In conclusione possiamo scrivere l'equazione di FK:
    \[\begin{aligned}
	\partial_{t}P(x,y) = \left[-\partial_{x}A_x - \partial_{y}A_y + \frac{1}{2}(\partial^2_{x^2}+\partial^2_{y^2}) \right]P(x,y) 
    .\end{aligned}\]
    Possiamo allora calcolare $Z_x$ e $Z_y$ ($B$ è l'identità quindi $B^{-1} = B$):
    \[
	 Z_x = 2(f+By) \qquad
	 Z_y = -\frac{2y}{\tau}
    .\] 
    Per essere una divergenza $\vect{Z}$ deve rispettare la condizione del rotore:
    \[
	\text{rot}(\vect{Z}) = 0
    .\] 
    Tuttavia questa condizione non è soddisfatta:
    \[
        \partial_{y}Z_x = 2B \neq \partial_{x}Z_y = 0
    .\] 
    Quindi questo processo non soddisfa la condizione di potenziale.
    \begin{greenbox}{}
        I processi di Ornstein-Ulhenback non hanno una distribuzione di equilibrio.
    \end{greenbox}
    \noindent
    Questo non significa che integrando questa equazione numericamente non si possa ottenere niente, questo tipo di processi presenta una \textbf{distribuzione di quasi equilibrio} (approfondiremo più avanti).
\end{exmp}
\noindent
\begin{exmp}[OU modificato]
Prendiamo lo stesso moto analizzato nell'esempio precedente ed aggiungiamo il termine $Bxdt$ all'equazione per $y$:
    \[
        \begin{cases}
	    dx = \left(f(x) + By\right)dt + d\omega_2\\
	    dy = -\dfrac{1}{\tau} y dt + Bxdt + d\omega_1
        \end{cases}
    .\] 
In questo caso abbiamo che:
\[\begin{aligned}
	&A_x = f+By & \quad B_{xx} = 1\\
	&A_y = -\frac{1}{\tau}y + Bx &\quad B_{yy}=1
.\end{aligned}\]
Mentre la matrice $B$ non cambia:
    \[
        B = 
	\begin{pmatrix}
	    B_{xx} & B_{xy}\\
	    B_{yx} & B_{yy}
	\end{pmatrix}
	=
	\begin{pmatrix}
	    1 & 0\\
	    0 & 1
        \end{pmatrix}
    .\] 
    Ed anche l'equazione di FP resta della stessa forma:
    \[\begin{aligned}
	\partial_{t}P(x,y) = \left[-\partial_{x}A_x - \partial_{y}A_y + \frac{1}{2}(\partial^2_{x^2}+\partial^2_{y^2}) \right]P(x,y) 
    .\end{aligned}\]
    Quello che cambia è la forma di $A_y$, che contiene il termine $Bx$ in più rispetto a prima. Questo comporta anche dei valori delle componenti di $\vect{Z}$ diversi:
    \[
	Z_x = 2(f+By) \qquad Z_y = -\frac{2y}{\tau}+2Bx
    .\] 
    Il rotore di $\vect{Z}$ è nullo grazie al termine aggiuntivo.\\
    Il termine aggiunto è quindi un termine ad-hoc per rendere il metodo visto sopra applicabile, infatti in questo modo è possibile avere una soluzione potenziale.
    \[
        P_{st}\sim \exp(\phi )
    .\] 
    con 
    \[\begin{aligned}
	\phi =& - \int\limits^{\vect{x}} d\vect{s}\vect{Z}  = \\
	      &= - \int\limits^{\vect{x}} (f+By)dx + \left(-\frac{y}{\tau} + Bx\right)dy = \\
	      &= \left(-\int\limits^{\vect{x}} fdx \right)-Bxy + \frac{y^2}{2\tau}
    .\end{aligned}\]
    Ricordiamo che gli integrali con pedice $\vect{x}$ sono integrali di linea, sul percorso $\vect{x}$ ad intervalli $d\vect{s}$.\\
    Questa cosa ha anche un senso fisico, infatti le equazioni differenziali per $x$ e $y$ possono essere riscritte come:
    \[\begin{aligned}
	&\frac{\text{d} x}{\text{d} t} = - \partial_{x}\left(-\int\limits^{\vect{x}} fdx - Bxy + \frac{y^2}{2\tau}\right) + \xi_x\\
	&\frac{\text{d} y}{\text{d} t} = - \partial_{y}\left(-\int\limits^{\vect{x}} fdx - Bxy + \frac{y^2}{2\tau}\right) + \xi_y
    .\end{aligned}\]
    Poiché svolgendo le derivate alcuni termini si annullano e tornano le due equazioni sopra ($xi = d\omega /dt$).\\
    Quindi le due equazioni si presentano in termini di equazioni del moto classiche, nella quale il termine che viene derivato rispetto alla coordinata in ciascuna equazione funge da potenziale.
\end{exmp}
\noindent
\subsection{Bilancio dettagliato}%
\label{sub:Bilancio dettagliato}
Il principio del bilancio dettagliato detta la struttura della equazione di FP e permetterà di risolverla in certe condizioni.\\
Il fatto che la corrente sia nulla e che si possa costruire un potenziale equivale a dire che il sistema presenta il bilancio dettagliato.
\subsubsection{Sistema di particelle in moto}%
\label{subsub:Sistema di particelle in moto}
Supponiamo di avere un sistema di oggetti caratterizzati da una posizione ed una  velocità ad un certo istante $(r,v,t)$.
La probabilità di trovarsi in $(r,v,t)$ e ad un altro istante in $(r',v',t')$ può essere scritta come probabilità congiunta:
\[
    P(r',v',t+\tau; r,v,t) \qquad t' = t+\tau
.\] 
Per trovare questa quantità possiamo ragionare in termini di propagatore (come abbiamo fatto nelle lezioni precedenti).
\[
    (r,v,t) \to (r',v',t+\tau) 
.\] 
\subsubsection{Metodo "Nolandiano"}%
\label{subsub:Metodo "Nolandiano"}
Possiamo chiederci quale sia la probabilità del processo inverso.
\[
    (r',v',t+\tau) \to (r,v,t') 
.\] 
Anziché pensare ad una inversione temporale per risalire ad un istante precedente possiamo decidere di invertire puntualmente la velocità. In questo modo manteniamo l'ordine temporale mentre gli oggetti si muovono all'indietro (vedi Tenet)
\[
    (r',-v',t) \to (r,-v,t+\tau) 
.\] 
\begin{redbox}{Conseguenza del Bilancio dettagliato}
    Se il sistema presenta bilancio dettagliato allora vale che:
    \[
    P(r',v',t+\tau; r,v,t) = P(r,-v,t+\tau; r',-v',t)
    .\] 
\end{redbox}
\noindent
Ovviamente in un caso stazionario avviamo anche l'invarianza per traslazione temporale, quindi possiamo considerare l'istante $t=0$:
\[
    P_s(r',v',\tau; r,v,0) = P_s(r,-v,\tau; r',-v',0)
.\] 
Ipotizziamo adesso che il processo sia Markoviano. In tal caso la probabilità composta si può esprimere in termini della distribuzione stazionaria $P_s(r,v)$ e del propagatore:
\[\begin{aligned}
    P&(r',v',\tau|r,v,0) P_s(r,v) =\\
     & =P(r,-v,\tau|r',-v',0) P_s(r',-v')
.\end{aligned}\]
Possiamo manipolare questa espressione valutando quali variabili cambiano segno sotto inversione temporale, in generale è vero che:
\begin{itemize}
    \item Le variabili di tipo "coordinate" non cambiano segno
	\[
	    t\to -t \implies  x \to x
	.\] 
    \item Le variabili di tipo velocità cambiano segno
	\[
	    t\to -t \implies  v_x \to -v_x
	.\] 
\end{itemize}
Prendendo delle coordinate generalizzate $x_i$ abbiamo che per inversione temporale:
\[
    x_i \xrightarrow[]{t\to -t} \epsilon_i x_i \qquad \epsilon_i = \pm 1
.\] 
Quindi ragionando in termini vettoriali:
\[
    \vect{x}  = (x,y,z,v_x,v_y,v_z) 
.\] 
La notazione usata allude alle coordinate di una singola particella, si può generalizzare una vettore $\vect{x}$ a $n$ particelle mettendo tutte le coordinate delle particelle $1,\ldots,n$ all'interno del vettore.\\
Possiamo riscrivere in modo generale i passaggi precedenti\footnote{Si è invertita la notazione degli indici primati\ldots}
:
\[
    P(\vect{x}, t+\tau ; \vect{x}', t) = 
    P(\vect{\epsilon}\cdot \vect{x}', t+\tau ; \vect{\epsilon}\cdot \vect{x}, t) 
.\] 
Prendiamo le condizioni "iniziali" per $\tau=0$:
\[\begin{aligned}
    \tau & = 0 :\\
	     &\implies  \delta (\vect{x}-\vect{x}') P_s(\vect{x}') = \delta (\vect{\epsilon} (\vect{x}'-\vect{x})) P_s(\vect{\epsilon}\vect{x}) 
.\end{aligned}\]
Sfruttando il fatto che la $\delta$ è pari e che la $\delta$ di una quantità vettoriale può essere vista come il prodotto di $\delta$ se ne conclude che:
\begin{redbox}{}
    \[\begin{aligned}
	P_s(\vect{x}') &= P_s(\vect{\epsilon}\cdot \vect{x}) \\
		       & \Downarrow\\
	P(\vect{x},t|\vect{x}', 0) P_s(\vect{x}') &=
        P\left(\vect{\epsilon}\vect{x}',\tau|\vect{\epsilon}\vect{x},0\right)P_s(\vect{x}') 
    .\end{aligned}\]
\end{redbox}
\noindent
La prima comporta che la variabile $v$ all'interno della $P_s$ dovrà comparire solo con potenze pari poiché $v$ cambia il suo segno quando si applica $\epsilon$.
\subsubsection{Conseguenze del bilancio dettagliato applicato ad una Chapman-Kolmogorov}%
\label{subsub:Conseguenze del bilancio dettagliato applicato ad una Chapman-Kolmogorov}
Si può dimostrare che (libro di Van Kampen) il bilancio dettagliato permette di descrivere la struttura generale dei vari termini della equazione di CK (equazione \ref{eq:4_CK}), le seguenti formule sono pura referenza ma ne faremo uso\ldots
\begin{equation} 
\begin{aligned}
    &\omega (\vect{x}|\vect{x}') P_s(\vect{x}') = \omega (\vect{\epsilon} \vect{x}'|\vect{\epsilon} \vect{x}) P_s(\vect{x}) \\
    &\epsilon_i A_i(\vect{\epsilon}\vect{x}) P_s(\vect{x}) =\left[-A_i(\vect{x}) + \frac{\partial }{\partial x_J} B_{iJ}(\vect{x})\right]P_s(\vect{x}) \\
    & \epsilon_i \epsilon_J B_{iJ}(\epsilon x) = B_{iJ}(\vect{x}) 
    \label{eq:11_ref}
.\end{aligned}\end{equation}
\subsubsection{Nota storica sul Bilancio dettagliato}%
\label{subsub:Nota storica sul Bilancio dettagliato}
Storicamente si sono introdotte due quantità per capire se il sistema presenta il bilancio dettagliato:
\[\begin{aligned}
    & D^{\text{irr}}_i = \frac{1}{2}\left(A_i(x) + \epsilon_i A_i(\epsilon_ix)\right)\\
    & D^{\text{rev}}_i = \frac{1}{2}\left(A_i(x) - \epsilon_iA_i(\epsilon_ix) \right)
.\end{aligned}\]
Infatti se la $D^{\text{irr}}_i\neq 0$ allora il sistema non presenta il bilancio dettagliato.

\begin{exmp}[]
    \[
        \begin{cases}
            dx = vdt\\
	    mdv = -V'(x) dt - \gamma v dt + \sqrt{2\gamma  k_B T} d\omega
        \end{cases}
    .\] 
    In cui si ha un termine di attrito alla Stokes:
    \[
        \gamma = 6\pi  \eta r
    .\] 
   L'equazione di FK che ne emerge è la seguente:
   \[\begin{aligned}
       &\partial_{t}P =\\
       &=-\partial_{x}v + \frac{1}{m}\partial_{v}\left[\left(V'(x) + \gamma v\right)P\right] +\\
       &  \ +\frac{\gamma k_b T}{m^2}\partial^2_{v^2}P
   .\end{aligned}\]
   Quindi abbiamo che:
   \[
       A = \begin{pmatrix} v \\ - \frac{\gamma  v}{m} - \frac{V'}{m}\end{pmatrix} 
       \qquad 
       B = \begin{pmatrix} 
	   0 & 0 \\
	   0 & \frac{2\gamma k_BT}{m^2}
           \end{pmatrix} 
   .\] 
   Visto che il termine di inversione $\vect{\epsilon}$ ci da:
   \[
       \vect{\epsilon}\cdot \begin{pmatrix} x \\ v \end{pmatrix} = \begin{pmatrix} x \\ -v \end{pmatrix} 
   .\] 
   Si ha anche che (si applica $\epsilon$ due volte, quindi dove opportuno c'è un doppio cambio di segno):
   \[
       \vect{\epsilon}\vect{ A}(\vect{\epsilon}\vect{x})   = \begin{pmatrix} -v \\ -\frac{\gamma v}{m} + \frac{V'}{m} \end{pmatrix} 
   .\] 
   Applicando adesso la seconda delle \ref{eq:11_ref}:
   \[\begin{aligned}
       \epsilon A(\epsilon x) P_s(x) &= - A(x) P_(x) + \partial_{J}B_{iJ}P_s(x) \\
				   & \Downarrow\\
       \begin{pmatrix} -v \\ -\frac{\gamma v}{m} + \frac{V'}{m} \end{pmatrix} P_s &= 
       \begin{pmatrix} -v \\ \frac{\gamma v}{m}+\frac{V'}{m} \end{pmatrix} P_s + 
       \begin{pmatrix} 0 \\ \partial_{v}\left(\frac{2\gamma k_BT}{m^2}\right) \end{pmatrix} P_s
   .\end{aligned}\]
   Notiamo che la derivata $J$ in questo caso indica la derivata sulla seconda componente: la velocità.\\
   La prima riga è una identità, la seconda riga invece è meno ovvia:
   \[
       -\frac{2\gamma v}{m}P_s = \frac{2\gamma k_BT}{m^2}\partial_{v}P_s
   .\] 
   Se risolviamo si ha una forma per la $P_s(x, v)$:
   \[
       P_s(x,v) = \exp\left(-\frac{v^2m}{2k_BT}\right)f(x) 
   .\] 
   Notiamo come la distribuzione stazionaria $P_s(x,v) $ sia quadratica in $v$.
   Questo, a conferma di quanto accennato prima, è dovuto all'invarianza di tale distribuzione quando si applica $\vect{\epsilon }$. \\
   Per trovare la $f(x) $ basta reinserire la $P_s$ nella equazione di FK,  si impone che il processo sia stazionario ($\partial_{t}P_s = 0$) e si ottiene:
   \[\begin{aligned}
       v\partial_{x}f(x) &= - \frac{v}{k_BT}V'(x) f(x) \\
			 &\Downarrow\\
       f(x) &\propto \exp\left(-\frac{V(x)}{k_BT}\right)
   .\end{aligned}\]
   In conclusione la distribuzione di equilibrio è la seguente:
   \[
       P_{s} = N \exp\left(-\frac{1}{k_BT}\left(\frac{mv^2}{2}+V(x)\right)\right)
   .\] 
   Notiamo come la forma sia esattamente quella che ci si aspetta da un caso fisico:
   \[
       P\sim \exp\left(-\beta H\right)
   .\] 
\end{exmp}
\noindent
Nell'esempio precedente si è legato il termine di dissipazione al termine stocastico (nelle SDE) nella formula:
\[
	    mdv = -V'(x) dt - \gamma v dt + \sqrt{2\gamma  k_B T} d\omega
.\] 
Notiamo infatti che $\gamma$ compare sia nel termine per $dt$ che in quello per $d\omega$.\\
Il motivo per il quale lo abbiamo fatto è dovuto ad un importante teorema.
\subsection{Teorema di Fluttuazione Dissipazione.}%
\label{sub:Relazione di Onsager e Teorema di Fluttuazione Dissipazione.}
Prendiamo un processo di OU nel quale all'interno della FK (di equazione \ref{eq:FK_general}) il termine $A$ è lineare ed il termine $B$ è costante
\[
    A_i(x) = A_{iJ}x_J \qquad
    B_{iJ}(x) = B_{iJ}
.\] 
Questa linearizzazione valuta le piccole oscillazioni attorno al punto di equilibrio per un processo di OU.\\
Se un processo ha le seguenti proprietà sotto inversione temporale
\footnote{Riscrivo le \ref{eq:11_ref} in un'altra forma, esplicitando il fatto che $B_{iJ}$ è costante.}
:
\begin{equation}
\begin{aligned}
    &\epsilon_i\epsilon_J B_{iJ}=B_{iJ}\\
    &\left(\epsilon_i\epsilon_JA_{iJ} + A_{iJ}\right)x_J = B_{iJ}\partial_{J}\ln P_s \label{eq:11_B}
\end{aligned}
\end{equation}
Allora è sempre vero che:
\[
    P_s = N \exp\left(-\frac{1}{2}x^tD^{-1}x\right)
.\] 
Con $D^{-1}$ simmetrica.
\paragraph{Relazioni di Onsager}%
\label{par:Relazioni di Onsager}
Se vale il principio del bilancio dettagliato allora esistono delle relazioni che legano la matrice $D^{-1}$ (che contiene l'equivalente della temperatura del sistema) ai vettori $A$:
\begin{equation}
\begin{aligned}
    &\epsilon A\epsilon  = D A^tD^{-1} \\
    &\epsilon A\epsilon D = D A^t \\
    &\epsilon (AD) = (AD)^t\epsilon\\
    & B = -\left(AD + DA^t\right) \label{eq:11_B1}
.\end{aligned}
\end{equation}
\begin{exmp}[Circuito RLC]
    Prendiamo il seguente circuito
    \begin{center}
\begin{circuitikz}[american, scale = 1.5][americanvoltages]
    \draw (0,0)
    to[] (0,2) 
    to[L, v^<=$L$] (1.5,2) %Inductor One  
    to[R, v^<=$R$] (3,2) % The resistor
    to[C, v^<=$C$] (3,0) %Capacitor Two
    to[V=$V \sim V_0 + \xi$] (0,0); % The voltage source
    \draw[thin, <-, >=triangle 45] (1.5,1)node{$i$}  ++(-60:0.5) arc (-60:170:0.5);
\end{circuitikz}
\end{center}

    Dove la sorgente di potenziale è una soluzione ionica (presenta delle fluttuazioni $\xi$).\\
    L'equazione che determina la corrente nel circuito è:
    \[
        \frac{\text{d} i}{\text{d} t} = \frac{1}{L}\left(-\frac{Q}{C}-iR +V_\xi\right)
    .\] 
    Supponendo che anche la carica sul condensatore senta delle fluttuazioni si ha che:
    \[
	\frac{\text{d} Q}{\text{d} t} = i - \gamma Q + Q_\xi(t) 
    .\] 
    Possiamo scrivere le SDE del processo stocastico come:
    \[
       \begin{cases}
 	 di = - \dfrac{Q}{LC}dt -\dfrac{R}{L}idt +\dfrac{V_\xi}{L}dt\\
	 dQ = idt - \gamma Q dt + Q_\xi dt          
       \end{cases} 
    .\] 
    La matrice $A$ si esprime come:
    \[
        A = 
	\begin{pmatrix} 
	    -\frac{R}{L} & -\frac{1}{LC}\\
	    1            & - \gamma
	\end{pmatrix} 
    .\] 
    Mi aspetto che la distribuzione stazionaria abbia una forma del tipo:
    \[
        P_s \sim \exp\left(-\frac{H}{k_BT}\right)
    .\] 
    L'energia del sistema sappiamo che vale:
    \[
        E = \frac{Li^2}{2} + \frac{Q^2}{2C}
    .\] 
    Guardando i termini dell'energia $\sim H$ ci aspettiamo una matrice $D$ della seguente forma
    \footnote{Dobbiamo considerare che l'inversa entrerà nella espressione per $P_s$}
    :
    \[
        D = 
	\begin{pmatrix}
	    \frac{k_BT}{L} & 0 \\
	    0              & k_BT C
	\end{pmatrix} 
    .\] 
    Mentre per quanto riguarda la matrice di inversione basta considerare l'analogia $(x,v) \to (Q,i)$:
    \[
	\epsilon  = \text{diag}(-1,1) 
    .\] 
    Possiamo verificare che siano rispettate le relazioni sulle matrici $A$ e $D$:
    \[
	AD = 
	\begin{pmatrix} 
	    \dfrac{Rk_BT}{L^2} & &  \dfrac{k_BT}{L}   \\
	     &&\\
	    - \dfrac{k_BT}{L} & & \gamma k_BTC 
	\end{pmatrix} 
    .\] 
    \[
	(AD)_{12} = -(DA)_{21} 
    .\] 
    Inoltre troviamo la matrice $B$ sfruttando l'ultima delle equazioni \ref{eq:11_B1}:
    \[
        B = -\left(AD+DA^t\right) = 2k_BT 
	\begin{pmatrix} 
	    R/L^2 & 0\\
	    0    & \gamma C
	\end{pmatrix} 
    .\] 
    Troviamo allora che tutto il meccanismo delle equazioni esposte sopra funziona in questo esempio.\\
    Quello che cerchiamo adesso è la relazione tra i termini stocastici nelle equazioni ($V_\xi$ e $Q_\xi$) e la temperatura. \\
    Si nota intanto che i due processi stocastici potrebbero essere legati tra loro, ipotizziamo che le variabili stocastiche che descrivono $V_\xi$ e $Q_\xi$ siano $\xi_1$ e $\xi_2$ tali che:
    \[
        \begin{cases}
            V_\xi /L \sim  b_{11} \xi_1 + b_{12}\xi_2\\
	    Q_\xi \sim b_{21}\xi_1 + b_{22}\xi_2
        \end{cases}
    .\] 
    Per avere tutte le formule di equilibrio che abbiamo ottenuto in precedenza la struttura della matrice $b$ deve essere (non ben spiegato perché): 
    \[
        b = 
	\begin{pmatrix} 
	    \sqrt{2k_BTR /L^2} & 0 \\
	    0             & \sqrt{2k_BT\gamma C} 
	\end{pmatrix} 
    .\] 
    Se ricordiamo bene le equazioni di partenza avevano le proprietà:
    \[
        \begin{cases}
            di \sim -\dfrac{R}{L}idt \\
	    dQ \sim - \gamma Qdt
        \end{cases}
    .\] 
    I due termini a moltiplicare $i$ e $Q$ sono proprio i termini che ritroviamo in $b_{11}$ e $b_{22}$.
\end{exmp}
\noindent
\begin{redbox}{Sunto del teorema Flutt. Diss.}
    Dato un processo con 
    \begin{itemize}
        \item Un termine di dissipazione $\gamma$
	\item Un processo stocastico $\xi$.
    \end{itemize}
    Allora l'intensità del processo stocastico vale $\sim \sqrt{2k_BT \gamma}$ (con normalizzazioni opportune)
\end{redbox}
\noindent
\subsection{FP come problema agli autovalori}%
\label{sub:Fokker-Plank dipendente dal tempo}
Riprendiamo l'equazione FK e cerchiamo una soluzione non stazionaria.
\begin{equation}
    \partial_{t}P = \left(-\partial_{x}A + \frac{1}{2}\partial^2_{x^2} B\right)P 
    \label{eq:11_FK_t}
\end{equation}
Possiamo vedere il termine nella parentesi di destra come un operatore: $\mathcal{L}$.
\[
    \partial_{t}P = \mathcal{L}P
\] 
che rappresenta una equazione agli autovalori. \\
In questa rappresentazione la $P$ si esprime come sovrapposizione lineare di autovalori $\lambda$ ed autovettori $P_\lambda$:
\begin{equation}
    P = P_\lambda (x) e^{-\lambda t}
    \label{eq:P_autoval}
\end{equation}
Le quantità $P_\lambda$ e $\lambda$ possono essere ottenute sostituendo nell'equazione \ref{eq:11_FK_t} l'espressione per la $P$ \ref{eq:P_autoval}.
\begin{equation}
    -\lambda P_\lambda  e^{-\lambda t} = \mathcal{L}P_\lambda  e^{-\lambda t}
    \label{eq:11_FK_auto}
\end{equation}
Possiamo utilizzare la notazione della meccanica quantistica pensando ai $P_\lambda$ nella equazione \ref{eq:11_FK_auto} come dei "ket". \\
Dobbiamo notare che l'equazione che agisce sui ket è nota, mentre quella che agisce sui bra no. Questo perché, a differenza del caso quantistico, qui c'è di mezzo la distribuzione stazionaria. \\
Quindi quando si cerca di applicare l'operatore $\mathcal{L}$ a sinistra si deve tener di conto che va ad agire anche sulla distribuzione stazionaria.\\
Dobbiamo quindi scrivere la $P(x,t)$ come:
\[
    P(x,t) = P_s(x) q(x,t) 
.\] 
Possiamo reinserire questa nella FK dipendente dal tempo e isolare i termini con $P_s$  tramite integrazione par parti. Ne emerge una equazione per $q(x,t)$, che rappresentano i bra, del tipo:
\begin{redbox}{Backward FK}
\[
    \partial_{t}q = A\partial_{x}q + \frac{1}{2}B\partial^2_{x^2}q
.\]     
\end{redbox}
\noindent
Che differisce dalla equazione per il ket dal solo segno del primo termine dopo l'uguale. \\
Abbiamo quindi il set di equazioni agli autovalori:
\begin{equation}
    \begin{cases}
	-\partial_{x}(AP_\lambda) + \dfrac{1}{2}\partial^2_{x^2}(BP_\lambda) = -\lambda P_\lambda\\
	\\
	A\partial_{x}Q_{\lambda'} + \dfrac{1}{2}B\partial^2_{x^2}Q_{\lambda'} = - \lambda'Q_{\lambda'}
    \end{cases}
\end{equation}
A questo punto si tratta di fare fare algebra sulle espressioni, possiamo moltiplicare la prima per $Q_{\lambda'}$, integrarla su un certo intervallo $(a,b)$  ed applicare l'integrazione per parti.\\
Il risultato al quale si arriva è il seguente:
\[\begin{aligned}
    (\lambda'&-\lambda) \int_{a}^{b} Q_{\lambda'}P_{\lambda}dx = \\
             &= \left\{Q_{\lambda'}\left[-AP_\lambda  + \frac{1}{2}\partial_{x}(BP_\lambda) \right] -
		       \frac{1}{2}BP_{\lambda}\partial_{x}Q_{\lambda'}\right\}_{a}^{b}
.\end{aligned}\]
Notiamo ancora l'importanza delle condizioni al contorno su $(a,b)$. 
\begin{exmp}[Condizioni al contorno assorbenti]
    In tal caso il lato destro della equazione si annulla in quanto:
    \[
	Q_{\lambda'}(a) = Q_{\lambda'}(b) = 0
    .\] 
    Quindi il sistema è "bi-ortogonale":
    \[
        \int Q_{\lambda'}P_{\lambda}dx = \delta_{\lambda\lambda'}
    .\] 
\end{exmp}
\noindent
\begin{exmp}[Processo di Wiener con condizioni al bordo ($0,1$) assorbenti.]
    Per via delle condizioni assorbenti in $0$ e $1$ abbiamo che:
    \[
	P(0,t) = P(1,t) = 0
    .\] 
    La FK che descrive il processo di Wiener è:
    \[
        \partial_{t}P = \frac{1}{2}\partial^2_{x^2}P
    .\] 
    Possiamo scegliere la base più semplice per lo sviluppo nell'intervallo che rispetti le condizioni al bordo:
    \[
	P_\lambda =\sin (\pi n x) 
    .\] 
    Quindi ogni $P$ si scrive come sovrapposizione:
    \[
	P(x,t) = \sum_{n=1}^{} b_n e^{-\lambda n t}\sin (\pi n x) 
    .\] 
    Inserendo nell'equazione per $P$  si ottengono gli autovalori:
    \[
        \lambda_n = \frac{n^2\pi^2}{2}
    .\] 
    Per quanto riguarda i coefficienti $b_n$  dobbiamo imporre delle condizioni iniziali:
    \[
	P(x,0) = \delta (x-x_0) 
    .\] 
    Quindi
    \[
	b_n = b_n(0) = \int_{0}^{1} \delta (x-x_0) \sin (n\pi x)  dx = \sin (n\pi x_0) 
    .\] 
    E con questo il problema è risolto, infatti si ha:
    \[
	P(x,t|x_0, 0) = \sum_{n=1}^{\infty} \sin (\pi nx_0) \sin (n\pi x) \exp\left(-\frac{n^2\pi^2 }{2}t\right)
    .\] 
\end{exmp}
\noindent
\begin{exmp}[Processo di Wiener in $(0,1)$ con condizioni riflettenti]
    Le condizioni riflettenti al bordo implicano che la corrente $J$ si annulla al bordo, vista l'espressione di $J$ (e ricordando che $A=0$ per il processo di Wiener) si ha:
    \[
	\frac{\partial P(0,t) }{\partial x} = \frac{\partial P(1,t) }{\partial x} = 0
    .\] 
    Quindi una base opportuna per le soluzioni è quella del coseno. Otteniamo quindi gli stessi autovalori del caso precedente:
    \[
        \lambda_n = \frac{n^2\pi^2}{2}
    .\] 
    Ma le autofunzioni sono adesso dei coseni, la soluzione finale è:
    \[
	P(x,t|x_0, 0) = \sum_{n=-\infty}^{\infty} \cos (\pi nx_0) \cos (n\pi x) \exp\left(-\frac{n^2\pi^2 }{2}t\right)
    .\] 
\end{exmp}
\noindent
\clearpage
