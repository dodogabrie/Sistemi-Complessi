\section{Lezione 9}%
\label{sub:Lezione 9}
\subsection{Oltre il teorema del limite centrale}%
\label{sub:Distribuzioni fisiche che non rispettano il teorema del limite centrale}
Abbiamo visto che, per variabili stocastiche con media e deviazione standard definite, vale:
\[
    S_n = \sum_{i}^{} x_i \to G
.\] 
Ci chiediamo se esista solo questa possibilità, ovvero se non esistano altre distribuzioni che possono far da limite per le variabili stocastiche.\\
Vediamo che succede se prendiamo una somma di variabili stocastiche estratte da una qualunque distribuzione:
\[
    S_2 = x_1 + x_2 \implies P_2(S_2) = \sum_{x_1}^{} P_1(x_1) P_1(x_2) 
.\] 
Si somma solo su $x_1$ poiché $x_2 = S_2-x_1$:
\[
    P_2(S_2) = \int P_1(x_1) P_1(S_2-x_1) dx_1
.\] 
Quindi si ha che:
\begin{redbox}{}
    \[
	P_2(S_2) = P_1(x_1) * P_1(x_2) 
    .\] 
\end{redbox}
\noindent
L'idea è che una distribuzione per essere stabile non deve cambiare sotto questa trasformazione (convoluzione).
\begin{exmp}[Distribuzione di Lorentz - Bright - Wigner]
\[
    P_L(x) = \frac{1}{\pi\left(1+x^2\right)}
.\] 
Le proprietà di questa distribuzione sono:
\[\begin{aligned}
    &\int P_L dx = 1\\
    & \int xP_L dx = 0\\
    & \int  x^2P_L dx \to \infty
.\end{aligned}\]
Per via della terza equazione non sono soddisfatte le condizioni del teorema del limite centrale.\\
Sviluppando la convoluzione si ha:
\[\begin{aligned}
    P_2(S_2) = \int P_1(x_1) P_1(S_2-x_1) dx_1 = \frac{1}{2\pi\left[1 + \left(\frac{S_2}{2}\right)^2\right]}
.\end{aligned}\]
La forma della distribuzione è rimasta invariata, l'unica differenza rispetto alla distribuzione di singola variabile è il fattore di scala $1 /2$.
\end{exmp}
\noindent
\begin{exmp}[Gaussiana]
\[
    P_1(x) = \frac{1}{\sqrt{2\pi}}\exp\left(-\frac{x^2}{2}\right)
.\]  
Valutando l'integrale per la $P_2$ si ha:
\[
    P_2(S_2) =\frac{1}{\sqrt{4\pi}}\exp\left(-\frac{(S_2)^2}{4}\right) 
.\] 
Abbiamo una struttura nuovamente Gaussiana con la $\sigma^2$ raddoppiata.
\end{exmp}
\noindent
\begin{exmp}[Distribuzione uniforme]
\[
    P_1(x) = 
    \begin{cases}
	1 & x \in \left[-\frac{1}{2}, \frac{1}{2}\right]\\
	0 & \text{Fuori}
    \end{cases}
\] 
In questo caso la forma della distribuzione cambia:
\[
    P_2(S_2) = \ldots = 1 - \left|S_2\right| 
.\] 
\end{exmp}
\noindent
Prendiamo adesso la somma di $n$ variabili sotto una certa distribuzione:
\[
    S_n = \sum_{i}^{n} x_i \implies  P(S_n) = P_1(x_1) * \ldots * P_n(x_n) 
.\] 
Nel caso di Lorentz e Gauss la procedura è generalizzabile:
\[\begin{aligned}
    &\text{Gauss}: \quad S_n = \frac{1}{\sqrt{n}} \sum_{i}^{n} x_i \implies  P_n(S_n) \in G\\
    &\text{Lorentz}: \quad S_n = \frac{1}{n} \sum_{i}^{n} x_i \implies  P_n(S_n) \in L
.\end{aligned}\]
\subsection{Distribuzioni stabili}%
\label{sub:Distribuzioni stabili}
\begin{bluebox}{Definizione di Distrib. Stabile}
    Una distribuzione si dice stabile se è invariante sotto convoluzione.
    \[\begin{aligned}
	P&(a_1z+b_1)*P(a_2z+b_2) =\\
	&=\int\limits_{-\infty}^{\infty} dy P(a_1(z-y) +b_1) P(a_2y + b_2)  = \\
	& = P(az + b) 
    .\end{aligned}\]
    con $a_i >0$, $b_i \in R$.
\end{bluebox}
\noindent
Andiamo in trasformata, chiamiamo la funzione caratteristica $P(k)$:
\[
    P(k) = \mathcal{F}\left[P(x) \right] = \int_{-\infty}^{\infty} e^{ikx}P(x) dx 
.\] 
\'E noto che per convoluzione si ha:
\[
    \mathcal{F}\left[P_1*P_2\right] = \mathcal{F}\left[P_1\right]\cdot \mathcal{F}\left[P_2\right]
.\] 
Quindi nel caso visto sopra:
\[
    S_n = \sum_{i}^{n} x_i \implies  P_n(k) = \left[P_1(k)\right]^{n}
.\] 
Nei due casi interessanti discussi nella sezione precedente la trasformata diventa:
\[\begin{aligned}
    &P_G(k) \sim e^{-\frac{k^2}{2}}\\
    &P_L(k) \sim e^{-\left|k\right|}
.\end{aligned}\]
In particolare nel caso della lorenziana:
\[
    P_2(k) = e^{-2\left|k\right|} \xrightarrow[]{\mathcal{F}^{-1}} \frac{1}{2\pi}\frac{1}{1+\left(\frac{x}{2}\right)^2}
.\] 
\subsection{Teorema di Levy-Kintchine}%
\label{sub:Distribuzioni limite}
\begin{bluebox}{Teorema limite$\iff$stabile}
    Una distribuzione di probabilità $L(x)$ può essere la distribuzione limite di $S_n = \sum_{i}^{} x_i$ se e solo se $L(x)$ è stabile.
\end{bluebox}
\noindent
Abbiamo dimostrato l'esistenza di due distribuzioni limite: Gauss e Lorentz.\\
Il teorema di Levy-Kintchine ( o della rappresentazione canonica ) serve a generalizzare la forma di una distribuzione limite.
\subsubsection{Il teorema LK}%
\label{subsub:Il teorema LK}
Data una distribuzione di probabilità $L_{\alpha,\beta}(x)$ con $\alpha,\beta$ parametri. Tale distribuzione è stabile se e solo se il logaritmo della sua funzione caratteristica
\[
    L_{\alpha,\beta  }(k) = \left<e^{ikx}\right> = \int_{-\infty}^{\infty} dx e^{ikx}L_{\alpha,\beta  }(x)  
.\] 
ha la forma generale:
\begin{redbox}{}
\[\begin{aligned}
    \ln&(L_{\alpha,\beta}(k)) =\\
       &=\begin{cases}
	   i\mu k-\gamma \left|k\right|^\alpha\left[1-i\beta  \frac{k}{\left|k\right|}\tan\left(\frac{\pi\alpha}{2}\right)\right]; & \alpha\neq 1\\
																   &\\
	   i\mu k-\gamma \left|k\right|\left[1+i\beta  \frac{k}{\left|k\right|} \frac{2}{\pi}\ln\left|k\right|\right]; &\alpha = 1
       \end{cases}
\end{aligned}\]    
Con $\alpha, \mu, \beta,\gamma  \in R$ e
\begin{itemize}
    \item $0\le \alpha\le 2$
    \item $-1\le \beta\le 1$ 
    \item $\gamma\ge 0$ 
\end{itemize}
\end{redbox}
\noindent
Il motivo dei vincoli per $\alpha$ e $\beta$ e che, antitrasformando per la distribuzione nello spazio reale, devono essere rispettate le proprietà delle distribuzioni.
\begin{exmp}[Gaussiana]
    Ritroviamo la Gaussiana se
    \begin{itemize}
        \item $\alpha =2$ 
	\item $\beta$ qualunque
	\item $\mu$ qualunque
	\item $\gamma  =1$ 
    \end{itemize}
\end{exmp}
\begin{exmp}[Lorentz]
    Ritroviamo la distribuzione di Lorentz se
    \begin{itemize}
        \item $\alpha =1$ 
	\item $\beta  = 0$ 
	\item $\mu  = 0$ 
	\item $\gamma  = 1$ 
    \end{itemize}
\end{exmp}
\noindent
\noindent
Gli indici $\alpha$ e $\beta$ definiscono la forma e le proprietà della distribuzione mentre $\mu$ e $\gamma$ sono fattori di scala.
\subsubsection{Caratteristiche del funzionale di LK}%
\label{subsub:Caratteristiche del funzionale di LK}
\paragraph{Il ruolo di $\alpha$}%
\label{par:Il ruolo di alpha }

Il parametro $\alpha$ controlla la forma della $L_{\alpha,\beta  }$ per valori $\left|x\right|\to \infty$. Ipotizziamo infatti di avere solo il termine:
\[
    \ln (L_{\alpha,\beta  }(k) ) \sim -\left|k\right|^\alpha \implies  L_{\alpha, \beta  }(k) \sim e^{-\left|k\right|^\alpha}
.\] 
Antitrasformando si ottiene che:
\[
    \int_{}^{} dk e^{-\left|k\right|^\alpha} e^{-kx} \xrightarrow[]{\left|x\right|\to \infty} \sim \frac{1}{\left|x\right|^{\alpha +1}}
.\] 
Quindi si ha che:
\begin{bluebox}{Andamento asintotico di $L_{\alpha,\beta  }(x)$}
   \[
       L_{\alpha,\beta  }(x) \sim \frac{1}{\left|x\right|^{\alpha +1}} \quad \left|x\right|\to \infty
   .\]  
   Con $0\le  \alpha \le 2$ 
\end{bluebox}
\noindent
Sfruttando questo andamento asintotico possiamo vedere per quali $\alpha$ convergono i momenti di ordine $\delta$ con $0<\delta <\alpha$:
\[\begin{aligned}
    \left<\left|x\right|^\delta\right> = & \int dx \left|x\right|^\delta L_{\alpha,\beta  }(x) =\\
					 & = \int  dx \frac{\left|x\right|^\delta}{\left|x\right|^{\alpha +1}}
.\end{aligned}\]
Per avere un momento di ordien $\delta$ finito l'integrale deve convergere.
Se ne conclude che 
\begin{itemize}
    \item per $\alpha <2$ la varianza non è definita
    \item per $\alpha <1$ non è definita nemmeno la media.
\end{itemize}
Studiando un fenomeno fisico quello che abbiamo sempre fatto è stato cercare una scala del problema, quindi il momento secondo. \\
Per tutte le distribuzioni di Levy (tranne la Gaussiana) il momento secondo non è definito, questo le rende distribuzioni controintuitive.
\paragraph{Il ruolo di $\beta  $}%
\label{par:Il ruolo di beta}
Il parametro $\beta  $   controlla la simmetria della distribuzione:
\begin{itemize}
    \item $\beta =0$ $\implies$ $L_{\alpha,\beta  }(x) $ simmetrica
    \item $\beta  = \pm 1$ $\implies$ $L_{\alpha,\beta  }$ molto asimmetrica al variare del parametro $\alpha$. 
\end{itemize}
Se $0<\alpha <1$ e $\beta =1$ allora il supporto della distribuzione è $\left[\mu,\infty\right)$.
\begin{exmp}[Lorentziana]
    $\beta =0$, $\alpha =1$ 
    \begin{figure}[H]
    \centering
    \begin{tikzpicture}
	\begin{axis}[
	    width=7cm,
	    height=4cm,
	    xmin= -3, xmax= 3,
	    ymin= 0, ymax = 0.4,
	    axis lines = middle,
	    x label style={at={(axis description cs:1,-0.01)},anchor=north},
	    y label style={at={(axis description cs:0.5,1)},anchor=south},
	    xlabel={$x$},
	    ylabel={$L_{1,0}(x) $ },
	    xtick={0},
	    ytick={0},
	    xticklabel={},
	    yticklabel={},
	    ]
	    \addplot[domain=-3:3, samples=500]{1/(pi*(1+x^2))};
	\end{axis}
    \end{tikzpicture}
    \caption{\scriptsize Distribuzione di Lorentz-Bright-Wigner.}
    \label{fig:lore}
\end{figure}

\end{exmp}
\noindent
\begin{exmp}[Levy-Smirnov]
    $\beta =1$, $\alpha =1 /2$, si ha in tal caso:
    \[
	L_{\alpha,\beta  }(x) \sim \left(\frac{\gamma^{1 /\alpha}}{2\pi}\right) 
	\frac{1}{\left(x-\mu\right)^{3 /2}} e^{-\frac{\gamma^{1 /\alpha}}{2\left(x-\mu\right)}} \Theta (\mu) 
    .\] 
    Con $\Theta (\mu) $ funzione di Heayside.
    \begin{figure}[H]
    \centering
    \begin{tikzpicture}
	\begin{axis}[
	    width=7cm,
	    height=4cm,
	    xmin= 0, xmax= 10,
	    ymin= 0, ymax = 0.2,
	    axis lines = middle,
	    x label style={at={(axis description cs:1,-0.01)},anchor=north},
	    y label style={at={(axis description cs:0,1)},anchor=south},
	    xlabel={$x$},
	    ylabel={$L_{1 /2,1  }(x) $ },
	    xtick={0, 1},
	    ytick={0},
	    xticklabel={$\mu$},
	    yticklabel={},
	    ]
	    \addplot[domain=0:10, samples=500]{1/(2*pi)^(1/2)*1/(x-1)^(3/2)*exp(-1/(x-1))};
	\end{axis}
    \end{tikzpicture}
    \caption{\scriptsize Distribuzione di Levy-Smirnov}
    \label{fig:levy}
\end{figure}

\end{exmp}
\noindent
\subsection{Bacino di attrazione di una distribuzione}%
\label{sub:Bacino di attrazione di una distribuzione}
Data una $P(x) $ possiamo vedere a quale $L_{\alpha,\beta  }(x)$ converge con il seguente teorema:
\begin{greenbox}{Teorema del bacino di attrazione}
    $P(x)$ appartiene al bacino di attrazione di una distribuzione stabile $L_{\alpha,\beta  }(x)$ con $0<\alpha <2$ se e solo se:
    \[
	P(x) \sim \frac{\alpha a^\alpha c_{\pm}}{\left|x\right|^{1+\alpha}} \qquad  x\to \pm\infty
    .\] 
\end{greenbox}
\noindent
Le costanti $c_{\pm}\ge 0$, $a>0$ sono legate ai coefficienti di $L_{\alpha,\beta  }(x)$ da relazioni semplici (che non si riescono a leggere negli appunti).\\
Notiamo che anche i momenti di $P(x)$ sono definiti per $\delta <\alpha$ esattamente come quelli di $L_{\alpha,\beta  }(x)$.  
\subsection{Cambio di scala}%
\label{sub:Cambio di scala}
\begin{redbox}{}
    Le distribuzioni di Levy sono Self-similari.
\end{redbox}
\noindent
Consideriamo una distribuzione di Levy e prendiamo nuovamente:
\[
    S_n = \sum_{i}^{n} x_i
.\] 
Un cambio di scala che lascia invariata la forma della distribuzione mandando $n\to \infty$ è:
\[
    \tilde{S}_N = \frac{1}{B_N}\sum_{i}^{N}x_i - A_N	
.\] 
Con
\[
    B_N = aN^{1 /\alpha} \qquad 
    A_N = 
    \begin{cases}
	0 & 0<\alpha <1\\
	\dfrac{N\left<x\right>}{B_N} & 1 \le \alpha\le 2
    \end{cases}
.\] 
Questo cambio di scala può essere utilizzato come riprova una volta individuata una espressione per la distribuzione di Levy: 
se aumentando i termini della sommatoria si ha che $\tilde{S}_N$ va in se stessa allora abbiamo indovinato i coefficienti.
\begin{exmp}[Gaussiana]
    \[
        B_N = N^{1 /2}\sigma \qquad A_N = \frac{N\left<x\right>}{B_N} 
    .\] 
\end{exmp}
\noindent
\subsubsection{"Perché troviamo sempre Gaussiane?"}%
\label{subsub:"Perché troviamo sempre Gaussiane?"}
Possiamo chiederci perché le Gaussiane spiccano in fama e diffusione nei processi fisici, la risposta arriva dal fatto che:
\[
    P(x) \sim \frac{1}{x^{3+\epsilon}} \implies P(x) \to G \quad \forall \epsilon >0
.\] 
Quindi tutte le distribuzioni aventi questa proprietà tendono a delle gaussiane. Quando questa condizione non è rispettata si ottiene una distribuzione di Levy.
\subsubsection{Probabilità di tornare nell'origine}%
\label{subsub:Probabilità di tornare nell'origine}
Possiamo estrarre informazioni sulla distribuzione di Levy di un processo andando a cercare la probabilità di tornare nell'origine. Prendiamo ad esempio:
\[
    L_{\alpha,0}(x) \implies  L_{\alpha,0}(k) \sim e^{-\gamma\left|k\right|^\alpha}
.\] 
Considerando l'ennesima iterata abbiamo visto che:
\[
    S_n \sim e^{-n\gamma\left|k\right|^\alpha}
.\] 
Adesso possiamo trovare la probabilità di rientrare in $x=0$  all'ennesima iterata semplicemente antitrasformando:
\[
    L^n_{\alpha,0}(0) = \frac{1}{\pi}\int_0^{\infty}\cos (0) e^{-n\gamma\left|k\right|^\alpha}dk = 
    \frac{\Gamma (1 / \alpha) }{\pi\alpha\left(\gamma n\right)^{1 / \alpha}}
.\] 
Quindi la probabilità di tornare nell'origine in funzione di $n$  scala come una potenza di $\alpha$. Questo ci permette di determinare il valore di $\alpha$.
\[
    L^n_{\alpha,0}(0) = \frac{P(S_n)}{n^{1 / \alpha}} 
.\] 
\subsection{SDE con variabili stocastiche con distrib. di Levy}%
\label{sub:SDE con variabili stocastiche con distrib. di Levy}
Prendiamo una SDE del tipo:
\[
    dx(s)  = dL_{\alpha,\beta  }(s) 
.\] 
\subsubsection{Integrale un processo di Levy}%
\label{subsub:Integrale un processo di Levy}
Definiamo l'integrale di una funzione $f(s)$ secondo un processo di Levy come:
\[\begin{aligned}
    &\int_{t_0}^{t} f(s) dL_{\alpha,\beta  }(s) = \\
    &=\lim_{N \to \infty} \sum_{i=1}^{N} f(\Delta s(i-1) ) M_{\alpha,\beta  }\left(\left[\Delta s(i-1), \Delta s(i)\right]\right)
.\end{aligned}\]
In cui si è usata l'abbreviazione:
\[
M_{\alpha,\beta}(x,y) = L_{\alpha,\beta}(x)-L_{\alpha,\beta}(y)
.\] 
Sostanzialmente è l'integrale di Ito.
\subsubsection{Algoritmo di Weron}%
\label{subsub:Algoritmo di Weron}
Per risolvere è necessario inventare una tecnologia per valutare le variabili stocastiche secondo una distribuzione di Levy generale. \\
L'algoritmo di Weron si occupa proprio di come generare questi numeri casuali $\xi$ (Vedi appendice), secondo tale algoritmo l'integrale di una funzione si può scrivere come:
\[
    \int_{t_0}^{t} f(s) dL_{\alpha,\beta  }(s) = \sum_{i}^{N} f(\Delta s(i-1)) \left(\Delta s\right)^{1 /\alpha}\xi_i
.\] 
Con $\Delta s = (t-t_0) /N$, ovvero l'ampiezza di ogni intervallo temporale, facendo una integrazione elementare si prende ad esempio $N=1$.\\
Come anticipato il numero $\xi_i$ è generato tramite una distribuzione di Levy con alcuni parametri:
\[
    \xi_i \in L_{\alpha,\beta  }\left(Z,\gamma =\frac{1}{\left(2\right)^{1 /2\alpha}}, \mu =0\right)
.\] 
Quindi operativamente si ha:
\[
    x_{n+1} = x_n + h ^{1 /\alpha}\xi_n
.\] 
\subsubsection{Che devo fare per generare la $\xi$?}%
Mentre per generare numeri gaussiani possiamo utilizzare una infinità di librerie per generare un numero secondo una distribuzione di Levy non è detto che vi sia un qualche pacchetto. \\
Quindi se non conosci Python devi riscrivere una funzione di 2 pagine per generare un set di numeri random, se conosci python invece esiste una libreria di Scipy che si occupa di generare numeri secondo una qualunque distribuzione di Levy.
\clearpage
