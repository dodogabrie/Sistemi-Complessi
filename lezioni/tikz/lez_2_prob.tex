\\
\tikzstyle{block} = [draw, fill=blue!20, rectangle, 
    minimum height=3em, minimum width=6em]
\tikzstyle{sum} = [draw, fill=blue!20, circle, node distance=1cm]
\tikzstyle{input} = [coordinate]
\tikzstyle{output} = [coordinate]
\tikzstyle{pinstyle} = [pin edge={to-,thin,black}]

% The block diagram code is probably more verbose than necessary
\begin{tikzpicture}[auto, node distance=2cm,>=latex']
    % We start by placing the blocks
    \node [block] (a) {$\left(x_1, \ldots , x_n\right)$};
    \node [block, right of=a, node distance=3cm] (b) {descritto};
    \node [block, right of=b, node distance=3cm] (yes) {Max-Boltz};
    % We draw an edge between the controller and system block to 
    % calculate the coordinate u. We need it to place the measurement block. 
    \draw [->] (a) -- node[name=u] {} (b);
    \node [output, right of=b] (output) {};


    \node [block, below of=a] (a1) {$\boldsymbol{x}$};
    \node [block, right of=a1, node distance=3cm] (b1) {$\in$ };
    \node [block, right of=b1, node distance=3cm] (yes1) {$A$ };
    % We draw an edge between the controller and system block to 
    % calculate the coordinate u. We need it to place the measurement block. 
    \draw [->] (a1) -- node[name=u] {} (b1);
    \node [output, right of=b1] (output1) {};



    % Once the nodes are placed, connecting them is easy. 
    \draw [->] (b) -- node [name=y] {}(output);
    \draw [->] (b1) -- node [name=y] {}(output1);
    \draw [->] (b) -- node [name=y] {}(b1);
    \draw [->] (a) -- node [name=y] {}(a1);
    \draw [->] (yes) -- node [name=y] {}(yes1);
\end{tikzpicture}
