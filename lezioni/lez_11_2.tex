\section{Tempo di primo passaggio}%
\mylocaltoc
\subsection{Introduzione al MFPT}%
\label{sub:Tempo di primo passaggio o MFPT}
Ipotizziamo di avere un fenomeno stocastico e di osservarne l'andamento temporale. Possiamo ipotizzare anche che questo fenomeno presenti dei picchi randomici in maniera irregolare. \\
\begin{redbox}{Tempo di primo passaggio}
    Dato un fenomeno stocastico che presenta delle anomalie ricorrenti nel tempo si definisce tempo di primo passaggio \textbf{MFPT} l'intervallo temporale medio che intercorre tra le anomalie.
\end{redbox}
\noindent
L'analisi di tale quantità ha molta importanza in vari campi di ricerca, ad esempio l'analisi sismica, l'analisi delle forti piogge e la dinamica neuronale.\\
Per calcolare il MFPT è necessario:
\begin{itemize}
    \item Creare un modello del fenomeno in termini stocastici.
    \item Derivare una qualche quantità del modello che ci permetta di calcolare il tempo medio tra gli eventi.
\end{itemize}
Ad esempio possiamo avere una certa distribuzione iniziale di oggetti (o camminatori):
\tikzmath{\s = 1;}
\begin{figure}[H]
    \centering
    \begin{tikzpicture}
	\begin{axis}[
	    width=7cm,
	    height=4cm,
	    xmin= -1, xmax= 7,
	    ymin= 0, ymax = 0.5,
	    axis lines = middle,
	    x label style={at={(axis description cs:1,-0.01)},anchor=north},
	    y label style={at={(axis description cs:0.15,1)},anchor=south},
	    xlabel={$x$},
	    ylabel={$P(x, t=0)$ },
	    xtick={-1, 0, 6},
	    xticklabels={$b$, $0$, $a$},
	    ytick={0},
	    yticklabel={$0$},
	    ]
	    \addplot[domain=-1:7, samples=500]{(1/(sqrt(2*pi*\s))*e^(-((x-2)/\s)^2)};
	\end{axis}
    \end{tikzpicture}
    \caption{\scriptsize Distribuzione iniziale di camminatori.}
    \label{fig:lore}
\end{figure}

\noindent
Ciascuno di questi camminatori si muove secondo l'equazione differenziale stocastica del modello. Possiamo chiederci quanto tempo impiegheranno questi a raggiungere il punto $a$. \\
Possiamo notare subito che il tempo di passaggio dipenderà dalle condizioni al bordo su $a$: per condizioni assorbenti tale tempo sarà maggiore (i camminatori spariscono in $a$), per condizioni riflettenti il tempo sarà minore (i camminatori rimbalzano ed hanno altri step, altre possibilità di raggiungere $a$).\\
Per procedere possiamo seguire i passaggi:
\begin{itemize}
    \item Si calcola la probabilità che la distribuzione non esca dal dominio (nell'esempio il domino era $[a,b]$).
    \item Si scrive una equazione differenziale per la probabilità.
    \item Si risolve l'equazione differenziale.
\end{itemize}
\subsection{MFTP in 1D}%
\label{sub:MFTP in 1D}
Prendiamo un ensemble di camminatori stocastici (sostanzialmente immaginiamo un processo di diffusione dei camminatori) nell'intervallo unidimensionale $a\le x \le b$ con condizioni al contorno assorbenti:
\[
    P\left(a,t|x,0\right) = P\left(b,t|x,0\right)=0 
.\] 
La probabilità di essere ancora all'intervallo al tempo $t$ se al tempo $t=0$ i camminatori si trovavano in $x$ è $G(x,t)$:
\[
    G(x,t) = \int_{a}^{b} P(x',t|x, 0) dx' 
.\] 
Si tratta sostanzialmente della probabilità condizionata di stare in un punto tra $a$ e $b$ al tempo $t$.\\
Sia $T$ il tempo di uscita del camminatore dal segmento, la probabilità che $T\ge t$ con $t$ arbitrario vale:
\[
    \text{Prob}(T\ge t) = \int_{a}^{b} P\left(x',t|x,0\right)dx' = G(x,t)  
.\] 
Poiché se al tempo $t$  il camminatore sta ancora dentro l'intervallo allora sicuramente il tempo di uscita è maggiore di $t$.\\
Cerchiamo l'equazione differenziale alla quale soddisfa l'oggetto del moto. \\
Il problema è che l'equazione di FP che abbiamo visto per ora ci dice come evolve il propagatore, quindi in questo caso coinvolge la variabile $x'$. 
Noi vorremmo invece lasciar libera la variabile di integrazione $x'$ e far agire la FK su $x$. \\
Ci viene in aiuto allora la \texttt{Backward Fokker Plank}, si riesce ad ottenere una equazione per la quantità $P\left(x,t|y,t'\right)$. Si riporta adesso l'equazione completa (sul Gardiner si trova il conto completo):
\begin{redbox}{Backward FK}
\[\begin{aligned}
     \partial_{t'}&P\left(x,t|y,t'\right) = \\
    = & - \sum_{}^{} A_i(y,t') \partial_{y_i}P\left(x,t|y,t'\right) +\\
      & - \frac{1}{2}\sum_{i,J}^{} B_{iJ}(y,t') \partial_{y_i}\partial_{y_J}P\left(x,t|y,t'\right) + \\
      & + \int dz \omega\left(z|y,t'\right)\left[P\left(x,t|y,t'\right)-P\left(x,t|z,t'\right)\right]
.\end{aligned}\]    
\end{redbox}
\noindent
L'ultimo membro della Backward FP riguarda la parte dei processi a salto che trascuriamo in questa analisi 1D:
\[\begin{aligned}
    \partial_{t'}P\left(x',t|x,t'\right) = & -A(x) \partial_{x}P\left(x',t|x,t'\right) +\\
					   &-\frac{1}{2}B(x) \partial^2_{x^2}P\left(x',t|x,t'\right)
.\end{aligned}\]
Possiamo notare che per processi omogenei nel tempo deve valere la proprietà (traslazione temporale):
\[
    P\left(x',t|x,0\right) = P\left(x',0|x, -t\right) 
.\] 
Quindi il termine a sinistra dell'uguale nella BFK si scrive come:
\[\begin{aligned}
    \partial_{t'}P\left(x',t|x,t'\right) =& -\partial_{t}P(x',t-t'|x,0) = \\
					  & = -\partial_{t''}P\left(x',t''|x,0\right)
.\end{aligned}\]
E l'equazione completa diventa ($t''\to t$):
\[\begin{aligned}
    \partial_{t}P\left(x',t|x,0\right) &= A(x) \partial_{x}P\left(x',t|x,0\right) + \\
				       & + \frac{1}{2}B(x) \partial^2_{x^2}P\left(x',t|x,0\right)
.\end{aligned}\]
Notiamo che la dipendenza temporale è stata spostata tutta sul termine "finale" del propagatore ($x',t$). Inoltre le derivate temporali sono applicate sul primo argomento del propagatore, quelle spaziali invece sul secondo argomento.\\
Integrando quest'ultima equazione tra $a$ e $b$ si ottiene una equazione per $G(x,t)$:
\begin{equation}
    \partial_{t}G(x,t) = A(x) \partial_{x}G(x,t) + \frac{1}{2}B(x) \partial^2_{x^2}G(x,t) 
    \label{eq:11_G_diff}
\end{equation}
Inserendo le solite condizioni iniziali: 
\[
    P(x',0|x,0) = \delta (x-x') \implies 
    G(x,0) = 
    \begin{cases}
	1 & a\le x\le b\\
	0 &\text{Altrove}
    \end{cases}
\] 
Inoltre se un camminatore raggiunge il bordo allora significa che la probabilità che esca ad un tempo superiore è nulla (poiché è appena uscito):
\[
    \text{Prob}(T\ge t) = 0 \quad \text{se} \quad x = a \text{ oppure } x = b
.\] 
Quindi anche:
\[
    G(a,t) = G(b,t) = 0
.\] 
Visto che il nostro insieme di camminatori, al passare del tempo, avrà una probabilità sempre maggiore di uscire dal segmento sarà vero che:
\[
    G(x, t+dt) < G(x,t)
.\] 
Il numero di camminatori usciti tra $t$ e $t+dt$ è proporzionale alla variazione di $G(x, t)$:
\[
    dG = G(x,t)-G(x,t+dt) = -\frac{\text{d} }{\text{d} t}(G(x,t)) dt
.\] 
Questa quantità ci permette di calcolare tutti i valori medi di funzioni dipendenti dal tempo in questo intervallo:
\[
    \left<f(t) \right>_x = -\int_{0}^{\infty} f(t) \frac{\text{d} }{\text{d} t} \left[G(x,t) \right]dt 
.\] 
In particolare il tempo medio di uscita, supponendo di essere in $x$  a $t=0$:
\begin{redbox}{MFPT}
\[
    T(x) \equiv \text{MFPT} = - \int_{0}^{\infty} t \partial_{t}G(x,t) dt 
.\]   
Integrando per parti:
\begin{align}
    T(x) &= \left. - t G(x, t) \right|_0^\infty + \int_{0}^{\infty} G(x,t) dt  = \nonumber\\
	 &= \int_{0}^{\infty} G(x,t) dt
    \label{eq:11_MFPT}
\end{align}
\end{redbox}
\noindent
In generale il "momento" n-esimo di primo passaggio vale:
\[
    T^n(x) = \left<T^n(x)\right> = \int_{0}^{\infty} t^{n-1}G(x,t) dt 
.\] 
Notando che si ha la relazione banale:
\[
    \int_{0}^{\infty} dt \partial_{t}G(x,t) = G(x,\infty) -G(x,0) = -1 
.\] 
Si possono sfruttare la \ref{eq:11_MFPT} e la \ref{eq:11_G_diff} per ricavare una equazione differenziale per il tempo di primo passaggio, infatti integrando nel tempo la \ref{eq:11_G_diff} si ottiene:
\[
    -1 = A\partial_xT(x) + \frac{1}{2}B \partial^2_{x^2}T(x) 
.\] 
L'equazione si risolve imponendo delle condizioni al contorno, ad esempio:
\[
    T(a) =T(b) =0
.\] 
La soluzione si ottiene in termini del seguente fattore integrante:
\[
    \phi (x) = \exp\left(\int_{a}^{x} \frac{2A}{B}dx' \right)
.\] 
che ci porta ad una equazione integrabile per $T(x)$:
\[
    \frac{\text{d} }{\text{d} x} \left[T' \phi (x) \right] = - \frac{2}{B}\phi (x) 
.\] 
Tale equazione si risolve integrando in $x$, successivamente dividendo per $\phi$\ldots La soluzione è:
\begin{bluebox}{Forma analitica di $T(x)$}
    Dato un intervallo $X = \left[a, b\right]$ (quindi con $a<b$) il tempo medio di primo passaggio si esprime come:
    \[
	T(x) = \frac{\Omega (x,b) - \Omega (a,x)}{\int_{a}^{b} \frac{dy}{\phi (y) }}
    .\] 
    In cui si definisce $\Omega$ come:
    \[\begin{aligned}
	\Omega &(x_1,x_2) = \\
	       & = \int_{a}^{x} \frac{dy}{\phi (y) }  \int_{x_1}^{x_2} dy'\left[\frac{1}{\phi (y') } \int_{a}^{y'} dz \frac{\phi (z)}{B(Z) } 
\right]    .\end{aligned}\]
\end{bluebox}
\noindent
Cambiando le condizioni al contorno cambia anche il risultato, i passaggi invece restano i medesimi.
\begin{exmp}[$a$ riflette e $b$ assorbe]
    Le condizioni ci dicono che:
    \[
	\left.\partial_{x}G(x,t) \right|_{a}=0
    .\] 
    E si può arrivare a:
    \begin{equation}
	T(x) = 2 \int_{x}^{b} \frac{dy}{\phi (y)} \int_{a}^{y} \frac{\phi (z) }{B(z) }  dz
	\label{eq:11_T_final}
    \end{equation}
\end{exmp}
\noindent
\subsection{MFPT per fuga da buca di potenziale}%
\label{sub:MFPT per fuga da buca di potenziale}
Prendiamo una buca di potenziale del seguente tipo:
\tikzmath{\a = 1; \b = 3;}
\usetikzlibrary{patterns}
\usepgfplotslibrary{fillbetween}

\begin{figure}[H]
    \centering
    \begin{tikzpicture}
	\begin{axis}[
	    width=7cm,
	    height=4cm,
	    xmin= -1, xmax= 5,
	    ymin= 0, ymax = 2.,
	    axis lines = middle,
	    x label style={at={(axis description cs:1,-0.01)},anchor=north},
	    y label style={at={(axis description cs:0.15,1)},anchor=south},
	    xlabel={$x$},
	    ylabel={$U(x)$ },
	    xtick={ 0, \a+0.1, 1.9},
	    xticklabels={$0$, $a$, $b$},
	    ytick={0},
	    yticklabel={$0$},
	    ]
	    \addplot[domain=0.2:5, samples=500]{((x-\a)^2+0.2)*(x-\b)^2 + 0.3};
	    \addplot[densely dotted, samples=50, smooth,domain=0:6, name path=one] coordinates {(\a+0.1,0)(\a+0.1,1.)};
	    \addplot[densely dotted, samples=50, smooth,domain=0:6, name path=two] coordinates {(1.9,0)(1.9,1.5)};
	\end{axis}
    \end{tikzpicture}
    \caption{\scriptsize Potenziale al quale sono soggetti i camminatori.}
    \label{fig:lore}
\end{figure}

\noindent
Ipotizziamo di preparare il sistema nell'intervallo tra minimo e massimo del potenziale $[ a, b ]$, l'equazione dell'evoluzione del propagatore sarà:
\begin{equation}
    \partial_{t}P = \partial_{x}\left(U'(x) P\right) + D\partial^2_{x^2}P
    \label{eq:MFTP_buca}
\end{equation}
In cui abbiamo cambiato notazione: $A(x) \to U'(x)$. Nella sezione precedente si è trovata la distribuzione di equilibrio della \ref{eq:MFTP_buca}:
\[
    P_s \approx N\exp\left[-\frac{U(x)}{D}\right]
.\] 
\tikzmath{\a = 1; \b = 3;}
\usetikzlibrary{patterns}
\usepgfplotslibrary{fillbetween}

\begin{figure}[H]
    \centering
    \begin{tikzpicture}
	\begin{axis}[
	    width=7cm,
	    height=4cm,
	    xmin= -1, xmax= 5,
	    ymin= 0, ymax = 1.8,
	    axis lines = middle,
	    x label style={at={(axis description cs:1,-0.01)},anchor=north},
	    y label style={at={(axis description cs:0.15,1)},anchor=south},
	    xlabel={$x$},
	    ylabel={$P_s(x)$ },
	    xtick={ 0, \a+0.1, 1.9},
	    xticklabels={$0$, $a$, $b$},
	    ytick={0},
	    yticklabel={$0$},
	    ]
	    \addplot[domain=0.2:5, samples=500]{e^(-((x-\a)^2+0.2)*(x-\b)^2 + 0.3)};
	    \addplot[densely dotted, samples=50, smooth,domain=0:6, name path=one] coordinates {(\a+0.1,0)(\a+0.1,0.65)};
	    \addplot[densely dotted, samples=50, smooth,domain=0:6, name path=two] coordinates {(1.9,0)(1.9,0.4)};
	\end{axis}
    \end{tikzpicture}
    \caption{\scriptsize Distribuzione di probabilità stazionaria.}
    \label{fig:lore}
\end{figure}

\noindent
A questo punto dobbiamo scegliere le condizioni al contorno su $a$ e $b$, prendiamo ad esempio le seguenti:
\begin{itemize}
    \item $b\equiv x_0$ bordo assorbente: le particelle che arrivano qui spariscono.
    \item $a\equiv \to -\infty$ come dire bordo riflettente poiché a $-\infty$ c'è un muro di potenziale divergente.
\end{itemize}
A questo punto possiamo prendere l'espressione \ref{eq:11_T_final} e specializzarla per il nostro problema. Si ottiene che il tempo di primo passaggio per andare da $a$ a $x_0$ vale:
\begin{align}
    &T(a\to x_0) = \nonumber\\
    & = \lim_{a \to -\infty} \frac{1}{D}\int_{a}^{x_0} dy\left[ e^{\frac{U(y)}{D}} \int_{a}^{y} e^{-\frac{U(z)}{D}}dz 
	\label{eq:exmp_T}
\right]
.\end{align}
Per risolvere analiticamente è comodo disaccoppiare i due integrali. 
Mettiamoci nel limite in cui la barriera di potenziale è molto maggiore del coefficiente di diffusione:
\[
    \Delta  U = U(b) -U(a); \qquad \frac{\Delta U}{D}\gg 1
.\] 
Concentrandoci in un intorno di $b$ possiamo notare che:
\[
    \exp\left(\frac{U}{D}\right) \text{ ha max in }b
.\] 
Inoltre ipotizzando che $U(x = a) = 0$ e se $x \to b$ si ha che:
\[
    -\frac{U(b)}{D} \to -\infty \implies  \exp\left(-\frac{U}{D}\right)\to 0
.\] 
Tale esponenziale decrescente è contenuto dell'integrale interno della \ref{eq:exmp_T}, allora si approssima tale integrale estendendone fino a $b$ l'estremo di integrazione. 
\[
 \int_{-\infty}^{y} \exp\left(-\frac{U(z)}{D}\right)dz \sim \int_{-\infty}^{b} \exp\left(-\frac{U(z)}{D}\right)dz   
.\] 
L'approssimazione consiste quindi nell'assumere che i termini della somma provenienti da un intorno di $b$ contino poco. I due integrali sono adesso disaccoppiati:
\[
    T \approx \frac{1}{D}\int_{-\infty}^{b}\exp\left(-\frac{U(z)}{D}\right)dz \int_{-\infty}^{x_0} dy \exp\left(\frac{U(y)}{D}\right)
.\] 
In conclusione si può risolvere per la $T(x)$ ipotizzando una forma parabolica per la $U(x)$ intorno a $a$ e $b$:
\[\begin{aligned}
    &U(x)  \approx U(b) - \frac{1}{2} \frac{\left(x-b\right)^2}{\delta^2} \ \text{  vicino a } b\\
    &U(x)  \approx U(a) + \frac{1}{2} \frac{\left(x-a\right)^2}{\alpha^2} \ \text{  vicino ad } a
.\end{aligned}\]
La soluzione prende la forma della famosa legge di Arrhenius:
\begin{greenbox}{Legge di Arrhenius}
\[
    T(a\to x_0) \approx 2\alpha\delta\pi  \exp\left(\frac{U(b) - U(a) }{D}\right)
.\] 
\'E una espressione simile alla legge di Arrhenius per le reazioni chimiche se poniamo $D = k_BT$.
\end{greenbox}
\noindent
\subsection{MFPT in più dimensioni}%
\label{sub:MFPT in più dimensioni}
Quando andiamo a studiare il caso multidimensionale si ha a che fare con questa equazione:
\begin{equation}
    \sum_{i}^{} A_i(x) \delta_iT(x)+\frac{1}{2}\sum_{i,J}^{} B_{iJ}\partial_{i}\partial_{J}T(x) = -1
    \label{eq:11_ mlti}
\end{equation}
Un modo elegante per risolvere è vederla come un problema agli autovalori. \\
Introduciamo il set di autofunzioni $Q_\lambda (x)$:
\[
    T(x) = \sum_{}^{} t_\lambda Q_\lambda(x) 
.\] 
Il problema si risolve reinserendo questa nella equazione \ref{eq:11_ mlti} e mettendo le opportune condizioni al contorno sulle $Q_\lambda$.\\
Procedendo in questo modo \ldots si può dimostrare che il tempo di primo passaggio prende la forma:
\[
    T(x) = \sum_{\lambda}^{} \frac{1}{\lambda}Q_\lambda (x) \int dx' P_\lambda (x')  
.\] 
Nei problemi tipici gli autovalori sono "separati esponenzialmente" l'uno dall'altro, quindi conta soltanto l'autovalore più basso (\ldots).\\
Il tempo di primo passaggio diventa quindi:
\[
    T(x) \approx \frac{1}{\lambda_1}Q_1\int P_1dx \approx \frac{1}{\lambda_1}
.\] 
\subsection{Calcolo numerico del MFTP}%
\label{sub:Calcolo numerico del MFTP}
Prendiamo la SDE per un set di camminatori:
\[
    dx = f(x) dt + g(x) d\omega
.\] 
Quindi per piccoli tempi possiamo scrivere che:
\[
    x_{n+1} = x_n + f(x_n) \Delta t + g(x_n) \Delta\omega_n
.\] 
Quindi mettiamoci in un punto $x_n$ e valutiamo la probabilità che $x_{n+1}$ sia fuori dal dominio considerato $(a,b)$. Ad esempio consideriamo la probabilità che $x_{n+1}$ sia oltre $b$.

Prendiamo il potenziale a doppia buca della sezione precedente, ipotizziamo che il camminatore elementare abbia fatto abbastanza passaggi da arrivare oltre il massimo $b$ e cadere nella seconda buca. \\
Possiamo chiederci quale sia la storia degli step effettuati da questo camminatore elementare, andando a vedere l'intensità del processo di Wiener in funzione della posizione si scopre che:
\begin{redbox}{}
Per superare il massimo del potenziale il processo di Wiener che spinge il camminatore deve essere sistematicamente diverso da zero.
\end{redbox}
\noindent
Ipotizziamo di avere l'andamento del processo stocastico per un camminatore $\omega (x) $, dimostriamo che tramite questo possiamo risalire al MFPT. 
\[
    x_{n+1} = x_n + f_n\Delta t + \sqrt{D}\Delta\omega_n
.\] 
Esplicitando tale relazione ricorsiva possiamo esprimere il passo $n$-esimo in funzione del passo $0$:
\[
    x_n = x_0 + \sum_{}^{} f_i \Delta t + \sqrt{D} \sum_{}^{} \Delta\omega_i
.\] 
Dalla prima possiamo estrarre $\Delta\omega_n$:
\[
    \Delta\omega_n = \frac{1}{\sqrt{D} }\left[x_{n+1}- \left(x_n + f_n \Delta t\right)\right]
.\] 
Sappiamo che la forma del processo di Wiener (la soluzione) è la seguente:
\[
P(\Delta\omega_n) \sim \exp\left(-\frac{\left(\Delta\omega_n\right)^2}{D\Delta t}\right)
.\] 
Per effettuare un salto da $a$ ad oltre il massimo $b$ abbiamo bisogno di una sequenza di salti giusti $\Delta\omega_i$, ovvero tali che:
\[
    x_0=a; \qquad x_n = b
.\] 
Quindi la probabilità di andare da $a$  a $b$ sarà la probabilità che tutti i processi di Wiener adeguati si verifichino:
\[
    P(a\to b) \sim \prod_{i}^{} \exp\left(-\frac{\left(\Delta\omega_i\right)^2}{D\Delta t}\right) 
.\] 
Ed inserendo la forma di $\Delta\omega_i$ ricavata in precedenza:
\[\begin{aligned}
    P(a\to b) &\sim \prod_{i}^{} \exp\left(-\frac{\left( x_{i+1}-(x_i + f_i\Delta t) \right)^2}{D\Delta t}\right) = \\
	      &= \exp\left(-\sum_{i}^{} \frac{\left( x_{i+1}-(x_i + f_i\Delta t) \right)^2}{D\Delta t}\right)=\\
	      &=\exp\left[-\sum_{i}^{} \frac{\Delta t}{D}\left(\frac{ x_{i+1}-x_i}{\Delta t} - f_i \right)^2\right] \to \\
	      & \to \exp\left[ -\frac{1}{D} \int dt \left(\dot{x} - f \right)^2\right]
.\end{aligned}\]
Per massimizzare la probabilità di passaggio da $a\to b$ è necessario minimizzare l'argomento dell'esponenziale, quindi si ricava il tempo minimo di primo passaggio tramite un problema di minimo:
\[
    \text{min}\int_{a}^{b} \left(\dot{x}-f\right)^2 dt
.\] 
Che rappresenta proprio la minimizzazione della Lagrangiana del sistema!\\
\clearpage
