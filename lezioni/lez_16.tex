\section{Lezione 16}%
\label{sub:Lezione 16}
\subsection{Trasformazioni canoniche}%
\label{sub:Trasformazioni canoniche}
Spesso nello studio dei sistemi Hamiltoninani è necessario trovare delle trasformazioni che permettano di integrare H. A tale scopo ci vengono incontro le trasformazioni canoniche.
\[
    H(p_i, q_i) = H'(P_i(p_i,q_i), Q_i(p_i, q_i) ) 
.\] 
Le nuove variabili rispettano le regole di commutazione canoniche:
\[
    \dot{Q}_i = - \left[ Q_i,H\right] \qquad \dot{P}_i = \left[P_i, H\right]
.\] 
Integrare l'Hamiltoniana significa trovare le $P_i, Q_i$  tali che:
\[
    H(p_i, q_i) \to H'(P_i) 
.\] 
Quindi per le equazioni di Hamilton-Jacoby anche:
\[\begin{aligned}
    & \dot{P}_i = - \frac{\partial H'}{\partial Q_i} =0\\
    & \dot{Q}_i = \frac{\partial H'}{\partial P_i} = f_i(P_i) 
.\end{aligned}\]
Questo ci permette di trovare una equazione del moto come:
\[
    \dot{Q}_i = f_i(P_i) t + Q_i(0) 
.\] 
Per ottenere una trasformazione canonica è necessario passare dalle \textbf{Funzioni generatrici}.
