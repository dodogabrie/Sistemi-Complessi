\section{Lezione 3}%
\label{sub:Lezione 3}
\subsection{Limiti con variabili stocastiche}%
\label{sub:Limiti con variabili stocastiche}
Un problema al limite con una variabile stocastica $x$ è un problema del seguente tipo:
\[
    x = \lim_{n \to \infty} x_n \qquad x_n \in \Omega
.\] 
Dove $x_n$ è una serie di funzioni di variabile stocastica $\omega$, indicizzata da $n$ intero. \\
La variabile stocastica $\omega$ appartiene all'insieme degli eventi possibili $\Omega$
\begin{defn}[\textcolor{red}{Almost Certain Limit}]
 \[
    \lim^{\text{ac}}_{n \to \infty} x_n = x
    \qquad \text{se } \forall \omega \in \Omega: \
    \lim_{n \to \infty} x_n(\omega) = x
.\] 
Tale limite rispetta la proprietà:
\[
    P(\lim_{n \to \infty} x_n(\omega) =x) =1
.\] 
\end{defn}
\begin{exmp}[Farfalle]
    Alcune specie di farfalle vivono mediamente un giorno, prendiamo tutte le farfalle viventi che appartengono a questa specie (ed escludiamo quelle che devono ancora nascere).\\
    Se misuriamo la media del cibo consumato da questo campione ogni ora otteniamo una sequenza di numeri casuali.
    Tuttavia possiamo essere quasi sicuri che tale sequenza dopo 30 ore sarà 0 e rimarrà 0 per sempre.\\
    In questo esempio si hanno: 
    \begin{itemize}
        \item $x_n$ la sequenza di medie del cibo.
	\item $\omega$ Il cibo che una farfalla della specie mangia in un'ora.
	\item $x = 0$.
    \end{itemize} 
\end{exmp}
\begin{exmp}[Equazione stocastica]
    Prendiamo una equazione stocastica differenziale così costruita:
    \[
        dx = - \alpha x dt + bxd\omega
    .\] 
    In cui il termine $d\omega$ è un termine stocastico.\\
    I passi degli $x_n$ saranno definiti dalla equazione sopra:
    \[
        x_{n+1} = x_n - \alpha  x_n \Delta t + b x_n \Delta\omega_n
    .\] 
    Se ipotizziamo che la $x_n$ (variabile composta da termine stocastico e termine temporale)  nell'evoluzione si avvicini all'origine ($x_n = 0$) allora il termine stocastico in $\omega$ non può più contribuire alla equazione, in tal caso la convergenza avviene.
\end{exmp}
\begin{defn}[\textcolor{red}{Mean Square Limit}]
    Se si ha:
    \[\begin{aligned}
	\lim_{n \to \infty} & 
	\left<\left(x_n(\omega) -x(\omega) \right)^2\right> =\\
			    &=\lim_{n \to \infty} 
			    \int  d\omega  P(\omega) 
			    \left[ x_n(\omega) -x(\omega) \right]^2 = 0
    .\end{aligned}\]
    Allora:
    \[
        \lim^{\text{ms}}_{n \to \infty} x_n = x
    .\] 
\end{defn}
\begin{defn}[\textcolor{red}{Limite in probabilità}] 
    \[
        \lim^{\text{P}}_{n \to \infty} x_n = x
    .\] 
    se vale che, $ \forall \epsilon  > 0$:
    \[
	\lim_{n \to \infty} P(\left|x_n-x\right| > \epsilon) = 0
    .\] 
\end{defn}
\noindent
Per capire cosa rappresenta il termine $P(\left|x_n - x\right|>\epsilon)$ introduciamo la funzione caratteristica:
\[
    \chi_\epsilon (t) = 
    \begin{cases}
	0 & \left|t\right|< \epsilon\\
	1 & \left|t\right|>\epsilon
    \end{cases}
.\] 

\usetikzlibrary{math}
\tikzmath{\x = 3; \y = 0;}
\begin{center}
\begin{tikzpicture}
     \draw[-stealth] (-\x,0) -- (\x,0) node[right]{$t$};
     \draw[-stealth] (0,0) -- (0,\x*2/3) node[above]{$\chi_\epsilon(t)$};
     \draw[thick] (-\x,\x/3) -- 
     		  (-\x/3, \x/3) -- 
		  (-\x/3, 0) node[below]{$-\epsilon$} --
		  (0,0) node[below]{$0$} -- 
		  (\x/3,0) node[below]{$\epsilon$} -- 
		  (\x/3,\x/3) -- 
		  (\x,\x/3);
\end{tikzpicture}
\end{center}
\noindent
Il termine di probabilità in questione è valutato proprio nel supporto di questa funzione:
\[
    P(\left|x_n-x\right|>\epsilon) = \int d\omega P(\omega) \chi_\epsilon (\left|x_n-x\right|) 
.\] 
\begin{defn}[\textcolor{red}{Limite in distribuzione}]
    Data una funzione di $x_n$: $f(x_n)$ si ha che questa converge a $f(x)$ in distribuzione se:
    \[
	\lim^{\text{dist}}_{n \to \infty}: \qquad \lim_{n \to \infty} \left<f(x_n)\right> = \left<f(x)\right>
    .\] 
\end{defn}
\begin{exmp}[Moto Browniano e moto a step]
    Prendiamo due moti con regole stocastiche differenti: un moto Browniano (passo del moto random) ed un moto casuale di passo unitario.
    Ipotizziamo che il corpo che effettua il moto (nei due casi) parta dall'origine.\\
    Le distribuzioni di probabilità dei fenomeni sono diverse, il valor medio del moto è invece nullo in entrambi i casi. Per questo motivo entrambi i moti tendono a 0 in distribuzione.
\end{exmp}
\noindent
Quest'ultimo limite è usato spesso "simulativamente", ovvero per la soluzione di equazioni differenziali stocastiche.\\
I limiti elencati in questa sezione non sono tutti indipendenti, infatti ($\implies$ = "implica" ) :
\[\begin{aligned}
    & \lim^{\text{ac}}_{n \to \infty} \implies  \lim^{\text{P}}_{n \to \infty} \\
    & \lim^{\text{ms}}_{n \to \infty} \implies  \lim^{\text{P}}_{n \to \infty} \\
    & \lim^{\text{P}}_{n \to \infty} \implies \lim^{\text{dist}}_{n \to \infty}  
.\end{aligned}\]


\subsection{Cumulanti della funzione caratteristica.}%
\label{sub:Sviluppo in cumulanti di phi}
\begin{redbox}{Funzione generatrice dei cumulanti}
   \[
       \Phi(s) = \ln (\phi (s) ) 
   .\]  
\end{redbox}
\noindent
Si potrebbe dimostrare che la funzione generatrice si esprime in modo generale in funzione di quantità definite come cumulanti:
\[\begin{aligned}
    \Phi = \sum_{r=1}^{\infty} i^r \sum_{\left\{m\right\}}^{} 
    \left<\left< x_1^{m_1}x_2^{m_2}\ldots\right>\right> 
    \prod_{i=1}^{\infty} \frac{s_i^{m_i}}{m_i!\,}\delta (r-\sum_{i=1}^{r} m_r) 
.\end{aligned}\]
Dove i termini tra le parentesi $\left<\left< x_i^{m_i}\right>\right>$ sono i cumulanti. Prendiamo ad esempio lo sviluppo dei primi due:
\[\begin{aligned}
    & \left<\left<x_1\right>\right> = \left<x_1\right> \sim \text{Media}\\
    & \left<\left<x_1 x_2 \right>\right> = \left<x_1x_2\right> - \left<x_1\right>\left<x_2\right> \sim \text{ Covarianza}
.\end{aligned}\]
Consideriamo adesso i cumulanti per una stessa variabile stocastica ($x_i = x_j $ $\forall i, j$), che chiameremo in questo contesto anche \texttt{Momenti}.
\begin{greenbox}{Cumulanti di processo Gaussiano.}
   I cumulanti per un processo Gaussiano sono tutti nulli per $n\ge 3$.
   \[
       \left<\left<x^n\right>\right> = 0 \quad \forall n \ge  3
   .\] 
\end{greenbox}
\begin{exmp}[Cumulante quarto per Gaussiana]
    \[\begin{aligned}
	\left<\left<x^4\right>\right> =& \left<x^4\right>-4\left<x^3\right>\left<x\right>+\\
					&-3\left<x^2\right>^2 + 12\left<x^2\right>\left<x\right>^2-6\left<x\right>^4
    .\end{aligned}\]
    Possiamo dimostrare che questo è nullo valutando la derivata del cumulante terzo:
    \[
        \int_{-\infty}^{\infty} \frac{\text{d} }{\text{d} x} \left\{x^3 \exp\left(\frac{-x^2}{2\sigma^2}\right)\right\} dx = 0 
    .\] 
    Questa si azzera perché la Gaussiana si annulla per $x = \pm \infty$, rendendo esplicita la derivata:
    \[
        \int\left(3x^2\exp\left(-\frac{x^2}{2\sigma^2}\right) - 
		\frac{x^4}{\sigma^2}\exp\left(- \frac{x^2}{2\sigma^2}\right) \right)dx = 0 
    .\] 
    \[
	\left<x^4\right> = 3\sigma^2\left<x^2\right> = 3 \left(\sigma^2\right)^2
    .\] 
    Inserendo nella equazione per il cumulante quarto si annullano tutti i termini (per semplicità abbiamo preso una gaussiana a media nulla $\left<x\right>=0$ ) .
\end{exmp}
\noindent
In generale questa cosa non funziona, non è possibile esprimere i cumulanti in funzione di altri cumulanti di ordine inferiore per ogni distribuzione.

\subsection{Teorema del limite centrale}%
\label{sub:Teorema del limite centrale}
\begin{redbox}{Teorema del limite centrale}
   La somma di variabili stocastiche aventi media e varianza definita tende ad una Gaussiana. 
\end{redbox}
\noindent
Sia $\left\{x_i\right\}$ una variabile random con distribuzione di probabilità $P_i(x_i)$ tale che:
\[\begin{aligned}
    & \left<x_i\right> = 0; &&
    & \text{var}\left\{x_i^2\right\} = b_i^2
.\end{aligned}\]
Il teorema richiede che i primi due momenti siano definiti:
\[\begin{aligned}
    & s_n = \sum_{i=1}^{n} x_i;
    &&
    &\sigma_n^2 = \sum_{i=1}^{n} b_i^2
.\end{aligned}\]
Inoltre le code della $s_n$ si devono annullare in modo rapido:
\[
    \lim_{n \to \infty} 
    \left[\frac{1}{\sigma_n^2} \sum_{i=1}^{n} \int\limits_{\left|x\right|>t\sigma_n}^{} dx x^2_i P_i(x)  \right] 
    = 0 \qquad \forall t>0
.\] 
Se ne conclude che 
\[
    \tilde{s}_n = s_n /\sigma_n \to G
.\] 
Con $G$ Gaussiana di media $0$ e varianza $1$.

\subsubsection{Teorema di Chebyshev}%
\label{subsub:Teorema di Chebyshev}
Si cerca di quantificare quanto velocemente una distribuzione tenda alla Gaussiana.\\
Definiamo la funzione:
\[
    F_n(t) \equiv \int_{-\infty}^{t} \tilde{P}(\tilde{s}_n)d\tilde{s}_n;  \qquad \phi (t) = \lim_{n \to \infty} F_n(t) 
.\] 
Dove $\tilde{P}$ è la distribuzione dei $\tilde{s}_n$, che tiene di conto che ad ogni step $n$ cambia la normalizzazione necessaria per essere una probabilità.
\begin{redbox}{Teorema di Chebyshev}
    \[
	F_n-\phi (t) \sim \frac{e^{-t^2 /2}}{\sqrt{2\pi}}\left[\frac{Q_1(t)}{n^{1 /2}} + \frac{Q_2(t)}{n} + \ldots\right]
    .\] 
    In cui i $Q_i$ sono i polinomi di Chebyshev-Hermite, legati ai momenti di $\left\{x_i\right\}$.
\end{redbox}
\noindent
Prendiamo ad esempio $Q_1(t)$:
\[
    Q_1(t) \propto \frac{\left<\left(x-\left<x\right>\right)^3\right>}{\sigma^3}
.\] 
la quantità a destra è legata al momento terzo di $\left\{x_i\right\}$, di conseguenza è nulla nel caso gaussiano (e lo sono anche tutte le restanti $Q_i$).\\
In conclusione le distribuzioni tendono ad una Gaussiana nelle ipotesi del teorema del limite centrale come $1 / \sqrt{n} $.

\subsection{Momenti fattoriali}%
\label{sub:Momenti fattoriali}

\subsubsection{Momenti fattoriali della distribuzione di Poisson}%
\label{subsub:Momenti fattoriali della distribuzione di Poisson}
\[
	P(x) = e^{-\lambda } \frac{\lambda^x}{x!\,}
.\] 
\begin{redbox}{Momenti fattoriali}
    \[
	\left<x^r\right>_f = \left<x\cdot (x-1) \cdot \ldots \cdot (x-r+1) \right>
    .\] 
\end{redbox}
\noindent
\begin{exmp}[Poisson con $r=2$.]
   \[\begin{aligned}
       \left<x\left(x-1\right)\right> =& \left<x\left(x-1\right)e^{-\lambda} \frac{\lambda^x}{x!\,}\right> = \\
       =& \lambda^2\left<e^{-\lambda} \frac{\lambda^{x-2}}{\left(x-2\right)!}\right> = \lambda^2
   .\end{aligned}\]
\end{exmp}
\noindent
Iterando questa procedura si ottiene:
\begin{bluebox}{Momenti fattoriali per distribuzione di Poisson}
 \[
    \left<x^r\right>_f = \lambda^r
.\]    
\end{bluebox}
\noindent

\subsubsection{Funzione generatrice generalizzata.}%
\label{subsub:Funzione generatrice generalizzata.}

\begin{redbox}{Funzione generatrice}
    \[
	G(s) = \sum_{n=0}^{\infty} s^n P(n) = \left<s^n\right>
    .\] 
\end{redbox}
\noindent
Possiamo ottenere la $G(s)$ a partire dalla funzione caratteristica:
\[
    G(s) = \phi (-i\ln s) 
.\] 
Grazie a questa possiamo esprimere i cumulanti fattoriali nel seguente modo:
\[
\left<x^n\right>_f = \left[\frac{\partial^n}{\partial s^n} G(s)\right]
.\] 
\begin{bluebox}{Funzione generatrice dei cumulanti fattoriali}
    \[
	g(s) \equiv \ln (G(s)) = \sum_{r=1}^{\infty} \left<\left<x^r\right>\right>_f \frac{\left(s-1\right)^r}{r!,}
    .\] 
\end{bluebox}
\noindent
\begin{exmp}[Funzione generatrice per Poissoniana]
    \[\begin{aligned}
	G(s) =& \sum_{n=1}^{\infty} s^n e^{-\lambda} \frac{\lambda^n}{n!\,} =\\
	=& e^{-\lambda}\sum_{n=1}^{\infty} \frac{\left(s\lambda\right)^n}{n!\,} =  e^{\lambda (s-1) }
     .\end{aligned}\]
     Per Poisson si ha quindi che:
     \[
         \left<\left<x^r\right>\right>_f = 0 \quad \forall r \ge  2
     .\] 
\end{exmp}
\noindent

\subsection{Processi stazionari e processi di Markov}%
\label{sub:Processi stazionari e processi di Markov}
\subsubsection{Probabilità di un processo}%
\label{subsub:Probabilità di un processo}
Prendiamo un oggetto vittima di un processo stocastico dipendente dal tempo e mettiamoci in un sistema di coordinate spaziali $x$. \\
Possiamo descrivere completamente il processo con:
\[
    P_n(x_1,t_1; x_2,t_2; \ldots ; x_n,t_n) 
.\] 
Ovvero la densità di probabilità che l'oggetto si trovi in $x_1$ al tempo $t_1$, $x_2$ al tempo $t_2$ etc\ldots \\
Scegliamo una base spazio temporale ($\overline{x} = (x,t)$), le proprietà di cui gode questa quantità sono:
\begin{itemize}
    \item $P_n \ge 0$.
    \item Simmetria: $P_n(\overline{x}_1;\overline{x}_2; \ldots) = P_n(\overline{x}_2; \overline{x}_1)$.
    \item Completezza:
	\[
	    \int P_n(\overline{x}_1;\ldots;\overline{x}_n) dx_n = P_{n-1}(\overline{x}_1;\ldots;\overline{x}_{n-1}) 
	.\] 
    \item Norma: $\int P_1(\overline{x}_1) dx_1 = 1$ 
\end{itemize}
Possiamo calcolare il valor medio di una quantità nel seguente modo:
\[\begin{aligned}
    &\left< x(t_1) \cdot \ldots \cdot x(t_n) \right> =\\
			     & \qquad=\int\limits_{R^n} dx_1\ldots dx_n P_n(\overline{x}_1;\ldots ;\overline{x}_n) x_1 \ldots x_n
.\end{aligned}\]


\begin{greenbox}{Processi stazionari}
    Un processo si dice stazionario se $\forall n$:
    \[\begin{aligned}
	P_n&(x_1,t_1;\ldots;x_n,t_n) =\\
	=&P_n(x_1,t_1+\Delta t;\ldots;x_n,t_n+\Delta t)
    .\end{aligned}\]
\end{greenbox}
\noindent
\subsubsection{Probabilità condizionata}%
\label{subsub:Probabilità condizionata}
Ipotizziamo che all'istante $t_i$ l'oggetto si trovi in $x \in \left[x_i, x_i + \Delta x\right]$ (con $i \in \left[1, k\right]$).\\
Allora la probabilità che l'oggetto si trovi in un istante successivo $t_{k+l}$ in un intervallo $\left[x_{k+l}, x_{k+l} + \Delta x\right]$ è:
\[\begin{aligned}
    P_{k+l}&(\overline{x}_1;\ldots; \overline{x}_{k+l}) =\\
    =&P_k(\overline{x}_1;..;\overline{x}_k)\cdot P_{l|k}(\overline{x}_{k+1};..;\overline{x}_{k+l}|\overline{x}_1;..;\overline{x}_k) 
.\end{aligned}\]
Con $P_{l|k}$ probabilità di trovarsi in un intorno di $x_{k+l}$ condizionata dai primi $k$ step. 
\begin{exmp}[Prob. condizionata dal primo step]
    \[
	P_2(\overline{x}_1;\overline{x}_2) = P_{1|1}(\overline{x}_2|\overline{x}_1) \cdot P_1(\overline{x}_1) 
    .\] 
\end{exmp}
\noindent

\begin{greenbox}{Processi di Markov}
    Un processo si dice Markoviano se:
    \[
	P_{1|n-1}(\overline{x}_n|\overline{x}_1;\ldots;\overline{x}_{n-1})  = P_{1|1}(\overline{x}_n|\overline{x}_{n-1}) 
    .\] 
\end{greenbox}
\noindent
Nel caso dei processi di Markov basta conoscere $P_1$ e $P_{1|1}$ per conoscere l'intero processo.

%%%%%%%%%%%%%%%%%%%%%%%%%%%%%%%%%%%%%%%
%  Equazione di Chapman - Kolmogorov  %
%%%%%%%%%%%%%%%%%%%%%%%%%%%%%%%%%%%%%%%

\subsection{Equazione di Chapman - Kolmogorov}%
\label{sub:Equazione di Chapman-Kolmogorov}

L'idea per l'equazione è data dall'identità:
\[
    \sum_{B}^{\Omega} P(A \cap B \cap C) = P(A \cap C) 
.\] 
Prendiamo un processo Markoviano, l'equazione per la $P_3$ è:
\[
    P_3(\overline{x}_1; \overline{x_2}; \overline{x}_3) = P_{1|1}(\overline{x}_3| \overline{x}_2) P_{1|1}(\overline{x}_2 | \overline{x}_1) 
    P_1(\overline{x}_1) 
.\] 
Integrando rispetto alla coord. $x_2$ otteniamo a sinistra la $P_2(\overline{x}_1;\overline{x}_3)$ (per la completezza delle $P_n$) , che può essere riscritta come:
\[
    P_2(\overline{x}_1;\overline{x}_3) = P_{1|1}(\overline{x}_3|\overline{x}_1) P_1(\overline{x}_1)
.\] 
Il risultato è l'equazione di Chapman-Kolmogorov.
\begin{redbox}{}
    \[\begin{aligned}
	P_{1|1}(x_3|x_1) = \int P_{1|1}(x_3|x_2) P_{1|1}(x_2|x_1)dx_2 \label{eq:3_CK}
    .\end{aligned}\]
\end{redbox}
\noindent
\clearpage
