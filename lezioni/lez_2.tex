\lez{2}{25-09-2020}{Argomento}
\subsection{Esempi di Master equation}%
\begin{itemize}
    \item Rumore shot.
    \item Rumore Jonshon
\end{itemize}
Prendiamo il rumore Shot, questo si basa sul fatto che la corrente non è proprio continua, chiamiamo $t_k$ il tempo di arrivo di un elettrone:
\[
    I(t) = \sum_{t_k}^{} F(t-t_k) 
.\] 
Mentre $F(t-t_k)$ è una funzione a pinna di squalo. Cerchiamo la Master equation per questo sistema:
\[
    P(n\to n+1, \text{in }\Delta  t) = \lambda \Delta tP_n(t) 
.\] 
Visto che possiamo riscrivere la probabilità di avere $n$ elettroni al tempo $t+\Delta t$ come: 
\[
    P_n(t+\Delta  t) = \left(1-\lambda\Delta t\right)P_n(t) + \lambda\Delta t 
.\] 
Si ottiene:
\[
    \frac{P(n, t+\Delta  t) - P(n,t)}{\Delta t} = \lambda (P(n-1,t) -P(n,t)) 
.\] 
\subsection{Metodo della funzione generatrice}%
Possiamo risolvere questa equazione utilizzando una tecnica standard: la funzione generatrice $G(s,t)$:
\[
    G(s,t) =\sum_{}^{} s^nP(n,t) 
.\] 
Sostituendo nella master si ha:
\[
    \frac{\partial G(s,t)}{\partial t} = \lambda (s-1) G(s,t) 
.\] 
Che si risolve con il risultato:
\[
    G(s,t) = \exp\left(\lambda (s-1) t\right)G(s,0) 
.\] 
Gli elettroni arrivano per $t\ge 0$, infatti si deve avere che: $P(0,0)=1$, $P(n,0) = 0 \ \forall n$, queste condizioni iniziali ci portano a $G(s,0) = 1$.
\[\begin{aligned}
    G(s,t) =& e^{\lambda (s-1) t}G(s,0) = \sum_{}^{} s^n P(n,t) \implies  \\
		&\sum_{}^{} e^{-\lambda t}\frac{\left(\lambda ts\right)^n }{n!}G(s,0)  = \sum_{}^{} s^nP(n,t) 
.\end{aligned}\]
In cui si è sfruttato la serie dell'esponenziale $e^{\lambda st}$.
\begin{redbox}{Distribuzione di Poisson}
    \[
	P(n,t)= e^{-\lambda t}\frac{\left(\lambda t\right)^{n}}{n!}
    .\] 
    Dove $P(n,t)$ è la probabilità che al tempo $t$ abbiamo $N(t) =n$ elettroni.
\end{redbox}
\noindent
Tornando alla corrente dobbiamo trovare un modo per contare gli elettroni:
\[
    \mu (t) = \frac{\text{d} N}{\text{d} t} \quad
    \begin{cases}
	&0 \quad \text{di solito}\\
	&\delta (t-t_k) \quad \text{Arriva e$^-$ al tempo }t_k
    \end{cases}
\] 
Dove ricordiamo che $t_k$ è random.
Quindi abbiamo che: 
\[
    \mu (t) = \sum_{k}^{} \delta (t-t_k) 
.\] 
Allora possiamo riscrivere la corrente con un integrale sfruttando le $\delta$:
\[
    I(t) = \int dx F(t-t_k) \mu (x) 
.\] 
Prendendo come $F$ modello la funzione:
\[
    F(t) = \begin{cases}
        0 \quad t<0\\
	q \exp\left(-\alpha t\right) \quad t \ge 0
    \end{cases}
.\] 
Otteniamo per la corrente:
\[
    I(t) = \int_{- \infty}^{t} q \exp\left(-\alpha (t-x) \right) \frac{\text{d} N}{\text{d} x} dx 
.\] 
Adesso dobbiamo risolvere il problema che $N(t)$ è una funzione a salti irregolari tra loro. Vediamo l'equazione differenziale per $I(t)$.
\[\begin{aligned}
    \frac{\text{d} I}{\text{d} t} =& q\exp\left(-\alpha (t-x)\right)\left.\dot{N}\right|_{x=t} + \\
				   &+\int_{-\infty}^{t} \left(-\alpha q\right)\exp\left(-\alpha (t-x)\right)\dot{N}dx 
.\end{aligned}\]
Risolvendo a sinistra ed usando le definizioni ci si riduce a:
\begin{redbox}{Equazione stocastica differenziale}
 \[
    \frac{\text{d} I}{\text{d} t} = -\alpha I(t) + q\mu (t) 
.\]    
\end{redbox}
Il termine in $\mu$ dipende dalla sequenza casuale di $\delta$, ogni sequenza casuale diversa ci può dare soluzioni diverse.\\
L'idea per risolvere il problema è di interpolare l'andamento di $N$ con un moto browniano, prendendo la media e le fluttuazioni del termine stocastico.
Essendo il termine in $\mu$ la derivata di un processo poissoniano abbiamo le seguenti proprietà:
\[\begin{aligned}
    &\left<\mu dt\right>=\left<dN\right> = \lambda dt\\
    & \left<\left(\lambda dt - \mu dt\right)^2\right> = \lambda dt
.\end{aligned}\]
Si ha un termine di fluttuazioni $d\eta$ tale che:
\[
    dN = \lambda dt + d\eta	
.\] 
Il differenziale della corrente si scrive come::
\[
    dI(t) = \left(\lambda q-\alpha I\right)dt+ qd\eta (t) 
.\] 
Prendendo la media di questa equazione abbiamo che il termine di fluttuazione media a zero:
\[
    \frac{\text{d} }{\text{d} t} \left<I\right> = \lambda q -\alpha\left<I\right>
.\] 
Questa equazione per tempi lunghi da il risultato stazionario:
\[
    \left<I\right>_{\infty}=\frac{\lambda q}{\alpha}
.\] 
Andiamo avanti nel conto con la seguente presa di posizione:
\begin{equation}
    \left(I+dI\right)^2 \approx I^2 + 2IdI	\label{eq:baddiff}
\end{equation}
Quindi con il risultato che dovrebbe esser noto sui differenziali: $d\left(I^2\right) = 2IdI$, se assumiamo questo e moltiplichiamo a destra e sinistra per $\left<I\right>$ nella equazione   per la corrente otteniamo:
\[
    \frac{1}{2}\frac{\text{d} }{\text{d} t} \left<I^2\right>= \lambda q\left<I\right>-\alpha\left<I^2\right>
.\] 
Che ci porta a concludere che:
\[
    \left<I^2\right>_{\infty} = \frac{\lambda q}{\alpha}\left<I\right>_{\infty} = \left(\left<I\right>_{\infty}\right)^2
.\] 
Otteniamo quindi un paradosso, la corrente ha varianza nulla:
\[
  \left<I^2\right>-\left<I\right>^2 = 0  
.\] 
Questo significherebbe che la "larghezza" del moto Browniano è nulla, quindi la corrente sarebbe costante e continua.\\
L'errore è dovuto al differenziale \ref{eq:baddiff}, infatti il termine trascurato vale:
\[
    \left<dI^2\right> = \left<q^2d\eta^2\right> = q^2\lambda dt
.\] 
Che è anch'esso di prim'ordine nel tempo! L'equazione corretta sarebbe allora:
\[
    \frac{1}{2}\frac{\text{d} }{\text{d} t} \left<I^2\right>= \lambda q\left<I\right>-\alpha\left<I^2\right> + q^2\lambda
.\]
