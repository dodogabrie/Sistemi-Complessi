\section{Lezione 2}%
\label{sub:Lezione 2}

\subsection{Esempi di Master equation}%
\begin{itemize}
    \item Rumore shot.
    \item Rumore Jonshon
\end{itemize}
\paragraph{Rumore shot}%
La corrente è composta da elettroni che si spostano, le fluttuazioni del numero di elettroni generano rumore.\\
Chiamiamo $t_k$ il tempo di arrivo di un elettrone su una certa sezione ortogonale alla direzione del flusso, si sceglie il seguente modello per la corrente:
\[
    I(t) = \sum_{t_k}^{} F(t-t_k) 
.\] 
$F(t-t_k)$ ha la forma di una scarica di condensatore:
\[
    F(t) = \begin{cases}
	0	 			&t<0\\
	q \exp\left(-\alpha t\right) 	&t \ge 0
    \end{cases}
\] 
Cerchiamo la Master equation per questo sistema:
\[
    P(n\to n+1, \text{in }\Delta  t) = \lambda \Delta tP_n(t) 
.\] 
Visto che possiamo riscrivere la probabilità di avere $n$ elettroni al tempo $t+\Delta t$ come: 
\[
    P_n(t+\Delta  t) = \left(1-\lambda\Delta t\right)P_n(t) + \lambda\Delta t  \implies 
\] 
\[
	\frac{P(n, t+\Delta  t) - P(n,t)}{\Delta t} = \lambda\left[P(n-1,t) -P(n,t)\right] 
.\] 
\subsection{Metodo della funzione generatrice}%
Per risolvere il problema introduciamo la funzione generatrice $G(s,t)$:
\[
    G(s,t) =\sum_{n}^{} s^nP(n,t) 
.\] 
Sostituendo nella master si ha:
\[
    \frac{\partial G(s,t)}{\partial t} = \lambda (s-1) G(s,t)
.\] 
\[
     \implies  G(s,t) = \exp\left(\lambda (s-1) t\right)G(s,0) 
.\] 
Gli elettroni arrivano per $t\ge 0$, infatti si deve avere che: 
\[
    \begin{cases}
	&P(0,0)=1\\
	&P(n,0) = 0
    \end{cases}
    \ \forall n \implies  G(s,0) = 1 
.\] 
\[\begin{aligned}
    G(s,t) =& e^{\lambda (s-1) t} = \sum_{}^{} s^n P(n,t) \implies  \\
		&\sum_{}^{} e^{-\lambda t}\frac{\left(\lambda ts\right)^n }{n!}  = \sum_{}^{} s^nP(n,t) 
.\end{aligned}\]
In cui si è sfruttato la serie dell'esponenziale $e^{\lambda st}$.
\begin{redbox}{Distribuzione di Poisson}
    \[
	P(n,t)= e^{-\lambda t}\frac{\left(\lambda t\right)^{n}}{n!}
    .\] 
    Dove $P(n,t)$ è la probabilità che al tempo $t$ ci siano $N(t) =n$ elettroni nel sistema.
\end{redbox}
\noindent
Tornando alla corrente dobbiamo trovare un modo per contare gli elettroni:
\[
    \mu(t) \equiv 
    \frac{\text{d} N}{\text{d} t} 
    =\delta (t-t_k)  
\] 
Dove $t_k$ è il tempo (random) in cui arriva un elettrone.
Passiamo al continuo nei tempi di arrivo degli elettroni e riscriviamo la corrente come:
\[\begin{aligned}
    I(t) =& \int dx F(t-x) \mu(x) =\\
	 =&\int_{- \infty}^{t} q \exp\left(-\alpha (t-x) \right) \frac{\text{d} N}{\text{d} x} dx 
.\end{aligned}\]
La difficoltà dell'espressione sta nel fatto che $N(t)$ è una funzione a salti irregolari.\\
Differenziando rispetto al tempo $I(t)$:
\[\begin{aligned}
    \frac{\text{d} I}{\text{d} t} &= q\exp\left(-\alpha (t-x)\right)\left.\dot{N}\right|_{x=t} + \\
				   &\qquad +\int_{-\infty}^{t} \left(-\alpha q\right)\exp\left(-\alpha (t-x)\right)\dot{N}dx =\\
				   &= q\mu (t) - \alpha I(t) 
.\end{aligned}\]
\begin{redbox}{Equazione stocastica differenziale}
 \[
    \frac{\text{d} I}{\text{d} t} = -\alpha I(t) + q\mu (t) 
.\]    
\end{redbox}
\noindent
Il termine in $\mu$ dipende dalla sequenza casuale di $\delta$, ognuna di queste può dare soluzioni differenti.\\
L'idea per risolvere il problema è di interpolare l'andamento di $N$ con un moto browniano, prendendo la media e le fluttuazioni del termine stocastico.
Essendo il termine in $\mu$ la derivata di un processo poissoniano abbiamo le seguenti proprietà:
\[\begin{aligned}
    &\left<\mu dt\right>=\left<dN\right> = \lambda dt\\
    & \left<\left(\lambda dt - \mu dt\right)^2\right> = \lambda dt
.\end{aligned}\]
Si ha un termine di fluttuazioni $d\eta$ tale che:
\[
    dN = \lambda dt + d\eta	
.\] 
Il differenziale della corrente si scrive come:
\[
    dI(t) = \left(\lambda q-\alpha I\right)dt+ qd\eta (t) 
.\] 
Prendendo la media di questa equazione abbiamo che il termine di fluttuazione $\left(I d\eta\right)$ media a zero per ipotesi:
\[
    \frac{\text{d} }{\text{d} t} \left<I\right> = \lambda q -\alpha\left<I\right>
.\] 
Il risultato stazionario per l'equazione è:
\[
    \left<I\right>_{\infty}=\frac{\lambda q}{\alpha}
.\] 
Procediamo con una ipotesi sbagliata: trascurare le fluttuazioni nel seguente termine
\begin{equation}
    \left(I+dI\right)^2 \approx I^2 + 2IdI	\label{eq:baddiff}
\end{equation}
Quindi con il risultato che dovrebbe esser noto sui differenziali: $d\left(I^2\right) = 2IdI$, se assumiamo questo e moltiplichiamo a destra e sinistra per $\left<I\right>$ nella equazione   per la corrente otteniamo:
\[
    \frac{1}{2}\frac{\text{d} }{\text{d} t} \left<I^2\right>= \lambda q\left<I\right>-\alpha\left<I^2\right>
.\] 
Che ci porta a concludere che:
\[
    \left<I^2\right>_{\infty} = \frac{\lambda q}{\alpha}\left<I\right>_{\infty} = \left(\left<I\right>_{\infty}\right)^2
.\] 
Otteniamo quindi un paradosso, la corrente ha varianza nulla:
\[
  \left<I^2\right>-\left<I\right>^2 = 0  
.\] 
Questo significherebbe che la "larghezza" del moto Browniano è nulla, quindi la corrente sarebbe costante e continua.\\
L'errore è dovuto al differenziale \ref{eq:baddiff}, infatti il termine trascurato vale:
\[
    \left<dI^2\right> = \left<q^2d\eta^2\right> = q^2\lambda dt
.\] 
Che è anch'esso di prim'ordine nel tempo! L'equazione corretta sarebbe allora:
\[
    \frac{1}{2}\frac{\text{d} }{\text{d} t} \left<I^2\right>= \lambda q\left<I\right>-\alpha\left<I^2\right> + q^2\lambda + O(dt)
.\]


\subsection{Correlazione e densità spettrale}%
\label{sub:Correlazione e densità spettrale}
Si rimembrano definizioni utili ai fini del corso. 
\begin{bluebox}{Correlazione}
    \[
        G\left(t\right)=\lim_{T \to \infty} \frac{1}{T}\int_{0}^{T} x\left(t+s\right) x\left(s\right)ds
    .\] 
\end{bluebox}
\noindent
Se il sistema è ergodico questa definizione è equivalente a quella mediata sull'Ensemble:
\[
    G( t) = \left<x(t+s) \cdot  x(s)\right>
.\] 
\begin{bluebox}{Densità spettrale}
    \[
        S\left(\omega\right) = \lim_{T \to \infty} \frac{1}{2\pi T}\left|\int_{0}^{T} x\left(t\right)e^{-i\omega t}dt \right|
    .\] 
\end{bluebox}
\noindent
Le due definizioni sono legate da una trasformata di Fourier: 
\begin{bluebox}{}
    \[
        S\left(\omega\right) = \frac{1}{2\pi}\int G\left(\omega\right)e^{-i\omega t}dt
    .\] 
\end{bluebox}



\subsection{Probabilità}%
\label{sub:Probabilità}
Possiamo ripensare una definizione assiomatica della probabilità.
%\\
\tikzstyle{block} = [draw, fill=blue!20, rectangle, 
    minimum height=3em, minimum width=6em]
\tikzstyle{sum} = [draw, fill=blue!20, circle, node distance=1cm]
\tikzstyle{input} = [coordinate]
\tikzstyle{output} = [coordinate]
\tikzstyle{pinstyle} = [pin edge={to-,thin,black}]

% The block diagram code is probably more verbose than necessary
\begin{tikzpicture}[auto, node distance=2cm,>=latex']
    % We start by placing the blocks
    \node [block] (a) {$\left(x_1, \ldots , x_n\right)$};
    \node [block, right of=a, node distance=3cm] (b) {descritto};
    \node [block, right of=b, node distance=3cm] (yes) {Max-Boltz};
    % We draw an edge between the controller and system block to 
    % calculate the coordinate u. We need it to place the measurement block. 
    \draw [->] (a) -- node[name=u] {} (b);
    \node [output, right of=b] (output) {};


    \node [block, below of=a] (a1) {$\boldsymbol{x}$};
    \node [block, right of=a1, node distance=3cm] (b1) {$\in$ };
    \node [block, right of=b1, node distance=3cm] (yes1) {$A$ };
    % We draw an edge between the controller and system block to 
    % calculate the coordinate u. We need it to place the measurement block. 
    \draw [->] (a1) -- node[name=u] {} (b1);
    \node [output, right of=b1] (output1) {};



    % Once the nodes are placed, connecting them is easy. 
    \draw [->] (b) -- node [name=y] {}(output);
    \draw [->] (b1) -- node [name=y] {}(output1);
    \draw [->] (b) -- node [name=y] {}(b1);
    \draw [->] (a) -- node [name=y] {}(a1);
    \draw [->] (yes) -- node [name=y] {}(yes1);
\end{tikzpicture}

Nella quale $\boldsymbol{x}$ è un evento, $A$ è un set di eventi possibili appartenente ad $\Omega$: l'insieme di tutti gli eventi possibili $(A \in \Omega)$.
\paragraph{Proprietà di $P$ }%
\label{par:Proprietà di P}
\begin{itemize}
    \item $P\left(A\right)\ge 0$.
    \item $P\left(\Omega\right) = 1$.
    \item $P\left(\cup_{i}^{} A_i \right) = \sum_{i}^{} P\left(A_i\right)$ con $A_i$ collezione di insiemi disgiunti numerabile.
    \item $P\left(\overline{A}\right) = 1- P\left(A\right)$.
    \item Se $\omega \in A \cup B$ con $A \cup B = 0$ $\implies$ $\omega\in A \ \cap \ \omega  \in B$.
\end{itemize}
\paragraph{Probabilità condizionata}%
\label{par:Probabilità condizionata}
La probabilità che avvenga l'evento $A$ sapendo che è già avvenuto un evento $B$ è data da:
\begin{greenbox}{}
	\[
	    P\left(A|B\right) = \frac{P\left(A \cap B\right)}{P\left(B\right)}
	.\] 
\end{greenbox}
Ovviamente si ha anche che:
\[
    P\left(A|B\right)P(B) = P(A)P\left(B|A\right)
.\] 
Prendiamo adesso un insieme $B_i$ di set di eventi:
\[\begin{aligned}
& B_i \cap B_j = \O \quad \forall i,j \\
& \bigcup_{i}^{} B_i = \Omega 
.\end{aligned}\]
Da queste ne deriva che, per il set $A$:
\[
    \begin{WithArrows}
	\bigcup_{i}^{}  \left(A \cap B_i \right) = A \cap \left(\bigcup_{i}^{}  B_i\right) = A \cap \Omega = A &
	\Arrow[jump=2,xoffset=0.5cm]{Utile per} \\
													       &\\
	\sum_{i}^{} P\left(A\cap B_i\right) = P\left(\bigcup_{i}^{} \left(A \cap B_i\right)\right) =  P(A) &
    \end{WithArrows}
.\] 
Per la probabilità congiunta si ha in generale che:
\begin{greenbox}{}
	\[
    		\sum_{i}^{} P\left(A|B_i\right)P(B_i) = \sum_{i}^{} P\left(A \cap B_i\right) = P(A) 
	.\]    
\end{greenbox}
\paragraph{Eventi indipendenti}%
\label{par:Eventi indipendenti}
Due eventi $x_A, x_B$ si dicono indipendenti se i set a cui appartengono $(A, B)$ rispettano la seguente:
\[
    P(B) P(A|B) = P(A\cap B) = P(A) P(B) 
.\] 
Questa fattorizzazione è valida in generale per eventi indipendenti:
\begin{bluebox}{}
\[
    P\left(\bigcap_i A_i\right) = \prod_i P(A_i) 
.\]    
\end{bluebox}
\paragraph{Valor medio e distribuzione di probabilità}%
\label{par:Valor medio e distribuzione di probabilità}
Sia $R$ una variabile random funzione di un evento $\omega$, il valor medio di $R$ sarà:
\begin{redbox}{}
 \[
    \left<R\right> = \sum_{\omega}^{} P(\omega) R(\omega) 
.\]    
\end{redbox}
\noindent
Possiamo facilmente estendere la definizione al caso continuo, definiamo il set $A(\omega_0, d\omega_0)$ come l'insieme degli eventi $\omega$ tali che:
\[
    \omega  \in \left[\omega_0, \omega_0 + d\omega_0\right]
.\] 
La densità di probabilità di trovare un evento nel set $A(\omega_0, d\omega_0)$ è data da:
\[
    P(\omega_0) d\omega_0 \equiv P\left[A(\omega_0,d\omega_0) \right] = P(\omega_0,d\omega_0) 
.\] 
Quindi il valor medio di $R$ nel continuo:
\begin{redbox}{Valor medio nel continuo}
 \[
    \left<R\right> = \int\limits_{\omega\in \Omega} R(\omega) P(\omega) 
.\]    
\end{redbox}
\subsection{Funzione caratteristica}%
\label{sub:Funzione caratteristica}
Sia $\vect{x}$ una variabile random con distribuzione di probabilità $P(\vect{x})$, la funzione caratteristica della distribuzione è:
\begin{redbox}{}
    \[
	\phi (\vect{s}) = \int d\vect{x} P(\vect{x}) e^{i \vect{x}\cdot \vect{s}}
    .\] 
\end{redbox}

\begin{exmp}[Distribuzione Gaussiana]
 \[\begin{aligned}
    &P(x) = \frac{e^{-x^2 / 2 \sigma^2}}{\sqrt{2\pi\sigma^2}} 
    &\implies&
    &\phi (s) = e^{-\sigma^2s^2 / 2}
.\end{aligned}\]
\end{exmp}

\begin{exmp}[Distribuzione uniforme]
\[
    P(x) = 
    \begin{cases}
	1 & x \in \left[-\frac{1}{2}, \frac{1}{2}\right]\\
	0 & \text{Altrimenti}
    \end{cases}
\] 
\[
    \phi (s) = \int_{-1 /2}^{1 /2} e^{ixs}dx = \frac{2}{s}\sin\left(\frac{s}{2}\right)  
.\] 
\end{exmp}

\subsubsection{Proprietà della funzione caratteristica}%
\label{ssub:Proprietà della funzione caratteristica}
\begin{enumerate}
    \item $\left|\phi (s) \right|= 1 $.
    \item $\phi (s) $ è continua.
    \item Se $\exists \left<x^n\right>$ allora: 
	\[
		\left<x^n\right> = \left(-i\right)^n \left.\frac{\partial ^n}{\partial s^n} \phi (s)\right|_{s=0} 
	.\] 
    \item Una sequenza di distribuzioni converge ad una distribuzione limite $\iff$ converge la sequenza di funzioni caratteristiche.
    \item Dato $\vect{x} = \left(x_1, \ldots, x_n\right)$ con $x_i$ indipendenti $\forall i$ allora:
	\[ 
	    \phi (s_1, s_2, \ldots) = \prod_{i=1}^{n} \phi (s_i)  
	\]
    \item Se $y = \sum_{i}^{} x_i$ con $x_i$ indipendenti, allora:
	\[
	    \phi (s) = \left<e^{isy}\right> = \prod_{i}^{} \phi_i(s)  
	.\] 
	\begin{exmp}
	    Se $y = x_1 + x_2$ ho che:
	    \[
		P(y) = \int P(x_1) P(y-x_1) dx_1
	    .\] 
	    Allora per le proprietà della trasformata di una convoluzione:
	    \[
		\phi (s) = \phi_1(s) \phi_2(s) 
	    .\] 
	\end{exmp}
\end{enumerate}
\clearpage
