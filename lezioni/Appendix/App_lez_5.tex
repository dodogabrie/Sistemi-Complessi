%%%%%%%%%%%%%%%%%%%%%%%%%%
%  Codice per appendice  %
%%%%%%%%%%%%%%%%%%%%%%%%%%
\noindent\rule{0.48\textwidth}{0.7pt}
\addtocounter{Sec}{\value{section}}%Mantengo il numero delle sezioni
\begin{appendices}
\section*{Appendice}%
\setcounter{section}{\theSec}%Applico numero sezioni
\setcounter{subsection}{0}%Applico numero sezioni
\renewcommand{\thesubsection}{\arabic{section}.\Alph{subsection}}

\subsection{Metodo delle Caratteristiche.}
\label{sub:caratteristiche}
Supponiamo di avere una PDE della forma:
\begin{redbox}{PDE per metodo delle caratteristiche}
\[
    a(x,y) \partial_{x}u + b(x,y) \partial_{y}u - c(x, y) = 0
.\] 
\end{redbox}
\noindent
Scrivibile anche come:
\begin{equation}
    \left(a,\ b,\ c\right)\cdot \left(\partial_{x}u,\ \partial_{y}u,\ -1\right) = 0
    \label{eq:caratt}
\end{equation}
Ed una superficie parametrizzata con la soluzione della PDE ($u(x,y))$: 
\[
    S \equiv (x,y, u(x,y) ) 
.\] 
\begin{center}
\begin{tikzpicture}[x={(170:.9cm)},y={(55:.6cm)},z={(90:1cm)}]
  %%%%%%%%%%%%%%%%%%%%
  %  Piano Tangente  %
  %%%%%%%%%%%%%%%%%%%%
  \tikzmath{\x = 2.2; \z = -0.4;}
  %%%%%%%%%%%%%%%%%
  %  Piano curvo  %
  %%%%%%%%%%%%%%%%%
  \draw[fill=red!75!black, opacity=0.5, looseness=.8] (\x,-\x,\z) node[above right] {$S$}
  to[bend left] (\x,\x,\z)
  to[bend left] coordinate (mp) (-\x,\x,\z)
  to[bend right] (-\x,-\x,\z)
  to[bend right] coordinate (mm) (\x,-\x,\z)
  -- cycle;
  %%%%%%%%%%
  %  Assi  %
  %%%%%%%%%%
  \draw[->] (0.5,0,-1.5) -- (-3,0,-1.5) node[right] {$y$};
  \draw[->] (0,0.4,-1.5) -- (0,-3,-1.5) node[right] {$x$};
  \draw[->] (0,0,-2) -- (0,0,1.5) node[right] {$u(x,y)$};
\end{tikzpicture}
\end{center}

\noindent
\subsubsection{Vettore tangente a $S$}%
\label{subsub:Vettore tangente a S}
\begin{bluebox}{}
Il vettore $\left(a, \ b, \ c\right)$ appartiene al piano tangente di $S$ in ogni punto $\left(x, y, z\right)$.
\end{bluebox}
\noindent
La normale $\vect{N}$ alla superficie $S$ la si trova facendo il gradiente di:
\[
    \overline{S} = u(x,y) - z
.\] 
Si ottiene quindi:
\[
    \vect{N}  = \left(\partial_{x}u, \ \partial_{y}u, \ -1\right)
.\] 
Visto che $\vect{N}$ è il secondo termine nella \ref{eq:caratt} si vede che la soluzione è il luogo dei vettori $(a, b,c)$ ortogonali a $\vect{N}$, quindi tangenti al piano $S$.
\begin{center}
\begin{tikzpicture}[x={(170:.9cm)},y={(55:.6cm)},z={(90:1cm)}]
  %%%%%%%%%%%%%%%%%%%%
  %  Piano Tangente  %
  %%%%%%%%%%%%%%%%%%%%
  \tikzmath{\x = 2.; \z = 0;}
  \draw[fill=black, opacity=0.3] (\x,-\x,\z) -- (\x,\x,\z) -- (-\x,\x,-\z) -- (-\x,-\x,-\z) -- cycle;
  %%%%%%%%%%%%%%%%%
  %  Piano curvo  %
  %%%%%%%%%%%%%%%%%
  \draw[fill=red!75!black, opacity=0.5, looseness=.8] (2.5,-2.5,-1) node[above right] {$S$}
  to[bend left] (2.5,2.5,-1)
  to[bend left] coordinate (mp) (-2.5,2.5,-1)
  to[bend right] (-2.5,-2.5,-1)
  to[bend right] coordinate (mm) (2.5,-2.5,-1)
  -- cycle;
  %%%%%%%%%%
  %  Assi  %
  %%%%%%%%%%
  \draw[->] (0,0,0) -- (0,-2.7,0) node[right] {$(a, b, c) $};
  \draw[->] (0,0,0) -- (0,0,2) node[right] {$N$};
\end{tikzpicture}
\end{center}

\noindent
Quindi la soluzione della PDE è tale per cui il vettore $(a,b,c)$ sta sul piano tangente.

\subsubsection{Curva caratteristica}%
\label{subsub:Curva caratteristica}
Per mappare la soluzione si introduce una curva $C$ detta curva caratteristica che descrive la superficie. 
\[
    C: \quad C \equiv \left(x(\eta) , y(\eta), z(\eta) \right)
.\] 
$C$ è una curva parametrica in $\eta$ localmente tangente a $(a, b, c)$.\\
La condizione di parallelismo implica il seguente sistema:
\begin{greenbox}{Equazioni Caratteristiche}
    Sono curve integrali per il campo vettoriale $(a, b, c)$ 
\[
    \begin{cases}
	 &a(x(\eta) , y(\eta) ) = \dfrac{\text{d} x}{\text{d} \eta} \\
								   &\\
	 &b(x(\eta) , y(\eta) ) = \dfrac{\text{d} y}{\text{d} \eta} \\ 
								   &\\
	 &c(x(\eta) , y(\eta) ) = \dfrac{\text{d} z}{\text{d} \eta}  
    \end{cases}
\]
\end{greenbox}
\noindent
Queste equazioni risolvono la PDE.
\begin{exmp}[Equazione del trasporto.]
   \[
       u_t + a \cdot u_x = 0
   .\]  
   In questo caso si ha $(a, b, c) \to (a, 1, 0)$, quindi:
   \[\begin{aligned}
	   &\dfrac{\text{d} x}{\text{d} \eta} = a & \quad
	   &\dfrac{\text{d} t}{\text{d} \eta} = 1 & \quad
	   &\dfrac{\text{d} z}{\text{d} \eta} = 0 &
   \end{aligned}\]
   Passiamo alla risoluzione:
   \[
       \begin{cases}
	    x(\eta) = a\eta +c_1\\
	    t(\eta) = c_2 + \eta\\
	    z(\eta) = c_3
       \end{cases}
       \implies\quad
       \begin{cases}
           x -at = x_0\\
	   z=k
       \end{cases}
   \] 
   In cui si è effettuata dell'algebra per eliminare $\eta$ nel primo sistema. 
   \begin{itemize}
       \item La funzione che risolve il sistema di destra è la soluzione dell'equazione del trasporto. 
       \item  Graficamente le funzioni che risolvono sono delle rette con $z$ costante, l'unione di queste rette rappresenta $S$.
       \item Abbiamo ottenuto un fascio di soluzioni poiché non abbiamo imposto alcuna soluzione al contorno.
   \end{itemize}
   In conclusione $z$ dovrà essere funzione di $x-at$, quindi la soluzione generale sarà una funzione del tipo:
   \[
       z(x, t ) = f(x-at) \equiv u(x, t) 
   .\] 
   Supponiamo che all'istante iniziale la soluzione fosse una gaussiana:
   \[
       f(x, t=0) = e^{-x^2}
   .\] 
   Quindi si ha che anche la soluzione a $t=0$ è una gaussiana:
   \[
       u(x, t=0) = e^{-x^2}
   .\] 
   Ed introducendo il tempo la soluzione diventa semplicemente:
   \[
       u(x, t) = e^{-\left(x-at\right)^2}
   .\] 
   
\end{exmp}
\noindent

\end{appendices}

\renewcommand{\thesubsection}{\arabic{section}.\arabic{subsection}}
%%%%%%%%%%%%%%%%%%%%%%%%%

