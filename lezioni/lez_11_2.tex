\section*{Parte 2}%
\label{sub:Parte 2}
\subsection{Tempo di primo passaggio o MFPT}%
\label{sub:Tempo di primo passaggio o MFPT}
Ipotizziamo di avere un fenomeno stocastico e di osservarne l'andamento temporale. Possiamo ipotizzare anche che questo fenomeno presenti dei picchi randomici in maniera irregolare. \\
In questa lezione cerchiamo un metodo analitico per esprimere l'intervallo temporale medio tra due eventi di questo tipo, anche detto tempo di primo passaggio.\\
Operativamente dobbiamo:
\begin{itemize}
    \item Creare un modello del fenomeno in termini stocastici.
    \item Derivare una qualche quantità dal modello che ci permetta di calcolare il tempo medio tra gli eventi.
\end{itemize}
Ad esempio possiamo avere una certa distribuzione iniziale di oggetti (o camminatori):
\tikzmath{\s = 1;}
\begin{figure}[H]
    \centering
    \begin{tikzpicture}
	\begin{axis}[
	    width=7cm,
	    height=4cm,
	    xmin= -1, xmax= 7,
	    ymin= 0, ymax = 0.5,
	    axis lines = middle,
	    x label style={at={(axis description cs:1,-0.01)},anchor=north},
	    y label style={at={(axis description cs:0.15,1)},anchor=south},
	    xlabel={$x$},
	    ylabel={$P(x, t=0)$ },
	    xtick={-1, 0, 6},
	    xticklabels={$b$, $0$, $a$},
	    ytick={0},
	    yticklabel={$0$},
	    ]
	    \addplot[domain=-1:7, samples=500]{(1/(sqrt(2*pi*\s))*e^(-((x-2)/\s)^2)};
	\end{axis}
    \end{tikzpicture}
    \caption{\scriptsize Distribuzione iniziale di camminatori.}
    \label{fig:lore}
\end{figure}

\noindent
Ciascuno di questi camminatori si muove secondo l'equazione differenziale stocastica del modello. Possiamo chiederci quanto tempo impiegheranno questi a raggiungere il punto $a$. \\
Possiamo notare subito che il tempo di passaggio dipenderà dalle condizioni al bordo su $a$: per condizioni assorbenti tale tempo sarà maggiore (i camminatori spariscono in $a$), per condizioni riflettenti il tempo sarà minore (i camminatori rimbalzano ed hanno altri step, altre possibilità di raggiungere $a$).\\
Per procedere possiamo seguire i passaggi:
\begin{itemize}
    \item Si calcola la probabilità che la distribuzione non esca dal dominio (nell'esempio il domino era $[a,b]$).
    \item Si scrive una equazione differenziale per la probabilità.
    \item Si risolve l'equazione differenziale.
\end{itemize}
\subsection{MFTP in 1D}%
\label{sub:MFTP in 1D}
Prendiamo un ensemble di camminatori stocastici (sostanzialmente immaginiamo un processo di diffusione dei camminatori) nell'intervallo unidimensionale $a\le x \le b$ con condizioni al contorno assorbenti:
\[
    P\left(a,t|x,0\right) = P\left(b,t|x,0\right)=0 
.\] 
La probabilità di essere ancora all'intervallo al tempo $t$ se al tempo $t=0$ i camminatori si trovavano in $x$ è $G(x,t)$:
\[
    G(x,t) = \int_{a}^{b} P(x',t|x, 0) dx' 
.\] 
Si tratta sostanzialmente della probabilità condizionata di stare in un punto tra $a$ e $b$ al tempo $t$.\\
Sia $T$ il tempo di uscita del camminatore dal segmento, la probabilità che $T\ge t$ con $t$ arbitrario vale:
\[
    \text{Prob}(T\ge t) = \int_{a}^{b} P\left(x',t|x,0\right)dx' = G(x,t)  
.\] 
Poiché se al tempo $t$  il camminatore sta ancora dentro l'intervallo allora sicuramente il tempo di uscita è maggiore di $t$.\\
Cerchiamo l'equazione differenziale alla quale soddisfa l'oggetto del moto. \\
Il problema è che l'equazione di FP che abbiamo visto per ora ci dice come evolve il propagatore, quindi in questo caso coinvolge la variabile $x'$. 
Noi vorremmo invece lasciar libera la variabile di integrazione $x'$ e far agire la FK su $x$. \\
Ci viene in aiuto allora la \texttt{Backward Fokker Plank}, si riesce ad ottenere una equazione per la quantità $P\left(x,t|y,t'\right)$. Si riporta adesso l'equazione completa (sul Gardiner si trova tutto il Ballino di conti):
\begin{redbox}{Backward FK}
\[\begin{aligned}
     \partial_{t'}&P\left(x,t|y,t'\right) = \\
    = & - \sum_{}^{} A_i(y,t') \partial_{y_i}P\left(x,t|y,t'\right) +\\
      & - \frac{1}{2}\sum_{i,J}^{} B_{iJ}(y,t') \partial_{y_i}\partial_{y_J}P\left(x,t|y,t'\right) + \\
      & + \int dz \omega\left(z|y,t'\right)\left[P\left(x,t|y,t'\right)-P\left(x,t|z,t'\right)\right]
.\end{aligned}\]    
\end{redbox}
\noindent
Tornando al problema MFPT in una dimensione l'equazione che ci serve è:
\[\begin{aligned}
    \partial_{t'}P\left(x',t|x,t'\right) = & -A(x) \partial_{x}P\left(x',t|x,t'\right) +\\
					   &-\frac{1}{2}B(x) \partial^2_{x^2}P\left(x',t|x,t'\right)
.\end{aligned}\]
Possiamo notare che per processi omogenei nel tempo deve valere la proprietà (traslazione temporale):
\[
    P\left(x',t|x,0\right) = P\left(x',0|x, -t\right) 
.\] 
Quindi il termine a sinistra dell'uguale nella BFK si scrive come:
\[\begin{aligned}
    \partial_{t'}P\left(x',t|x,t'\right) =& -\partial_{t}P(x',t-t'|x,0) = \\
					  & = -\partial_{t''}P\left(x',t''|x,0\right)
.\end{aligned}\]
E l'equazione completa diventa ($t''\to t$):
\[\begin{aligned}
    \partial_{t}P\left(x',t|x,0\right) &= A(x) \partial_{x}P\left(x',t|x,0\right) + \\
				       & + \frac{1}{2}B(x) \partial^2_{x^2}P\left(x',t|x,0\right)
.\end{aligned}\]
Notiamo che la dipendenza temporale è stata spostata tutta sul termine "finale" del propagatore ($x',t$). Inoltre le derivate temporali sono applicate sul primo argomento del propagatore, quelle spaziali invece sul secondo argomento.\\
Integrando quest'ultima equazione tra $a$ e $b$ si ottiene una equazione per $G(x,t)$:
\begin{equation}
    \partial_{t}G(x,t) = A(x) \partial_{x}G(x,t) + \frac{1}{2}B(x) \partial^2_{x^2}G(x,t) 
    \label{eq:11_G_diff}
\end{equation}
Inserendo le solite condizioni iniziali: 
\[
    P(x',0|x,0) = \delta (x-x') \implies 
    G(x,0) = 
    \begin{cases}
	1 & a\le x\le b\\
	0 &\text{Altrove}
    \end{cases}
\] 
Inoltre si deve avere anche che:
\[
    \text{Prob}(T\ge t) = 0 \quad \text{se} \quad x = a \text{ oppure } x = b
.\] 
Quindi anche:
\[
    G(a,t) = G(b,t) = 0
.\] 
Visto che il nostro insieme di camminatori, al passare del tempo, avrà una probabilità sempre maggiore di uscire dal segmento sarà vero che:
\[
    G(x, t+dt) < G(x,t)
.\] 
Il numero di camminatori usciti tra $t$ e $t+dt$ vale:
\[
    dG = G(x,t)-G(x,t+dt) = -\frac{\text{d} }{\text{d} t}(G(x,t)) dt
.\] 
Questa quantità ci permette di calcolare tutti i valori medi di funzioni dipendenti dal tempo in questo intervallo:
\[
    \left<f(t) \right>_x = \int_{0}^{\infty} f(t) \frac{\text{d} }{\text{d} t} \left[G(x,t) \right]dt 
.\] 
In particolare il tempo medio di uscita, supponendo di essere in $x$  a $t=0$:
\begin{redbox}{}
\[
    T(x) \equiv \text{MFPT} = - \int_{0}^{\infty} t \partial_{t}G(x,t) dt 
.\]   
Integrando per parti:
\begin{equation}
    T(x) = \int_{0}^{\infty} G(x,t) dt 
    \label{eq:11_MFPT}
\end{equation}
\end{redbox}
\noindent
In generale il "momento" n-esimo di primo passaggio vale:
\[
    T^n(x) = \left<T^n(x)\right> = \int_{0}^{\infty} t^{n-1}G(x,t) dt 
.\] 
Sfruttando la \ref{eq:11_MFPT} e la \ref{eq:11_G_diff} possiamo ricavare una equazione differenziale per il tempo di primo passaggio integrando e notando che:
\[
    \int_{0}^{\infty} \partial_{t}G(x,t) = G(x,\infty) -G(x,0) = -1 
.\] 
In conclusione:
\[
    -1 = AT'(x) + \frac{1}{2}BT''(x) 
.\] 
Con le condizioni al contorno banali:
\[
    T(a) =T(b) =0
.\] 
Possiamo risolvere l'equazione per $T(x)$ utilizzando il fattore integrante:
\[
    \phi (x) = \exp\left(\int_{a}^{x} \frac{2A}{B}dx' \right)
.\] 
che ci porta ad una equazione integrabile per $T(x)$:
\[
    \frac{\text{d} }{\text{d} x} \left[T' \phi (x) \right] = - \frac{2}{B}\phi (x) 
.\] 
In conclusione si ha:
\begin{bluebox}{Forma analitica di $T(x)$}
    \[
	T(x) = \frac{1}{N} \left[\Omega (x,b) - \Omega (a,x)\right]
    .\] 
    con 
    \[
	N = \int_{a}^{b} \frac{dy}{\phi (y) }
    .\] 
    Che funge da normalizzazione, mentre al numeratore abbiamo:
    \[\begin{aligned}
	\Omega &(x_1,x_2) = \\
	       & = \int_{a}^{x} \frac{dy}{\phi (y) }  \int_{x_1}^{x_2} dy'\left[\frac{1}{\phi (y') } \int_{a}^{y'} dz \frac{\phi (z)}{B(Z) } 
\right]    .\end{aligned}\]
\end{bluebox}
\noindent
Cambiando le condizioni al contorno cambia anche il risultato, anche se i passaggi concettuali restano i medesimi.
\begin{exmp}[$a$ riflette e $b$ assorbe]
    Le condizioni ci dicono che:
    \[
	\left.\partial_{x}G(x,t) \right|_{a}=0
    .\] 
    E si può arrivare a:
    \begin{equation}
	T(x) = 2 \int_{x}^{b} \frac{dy}{\phi (y)} \int_{a}^{y} \frac{\phi (z) }{B(z) }  dz
	\label{eq:11_T_final}
    \end{equation}
\end{exmp}
\noindent
\subsection{MFPT per fuga da buca di potenziale}%
\label{sub:MFPT per fuga da buca di potenziale}
Prendiamo una buca di potenziale del seguente tipo:
\tikzmath{\a = 1; \b = 3;}
\usetikzlibrary{patterns}
\usepgfplotslibrary{fillbetween}

\begin{figure}[H]
    \centering
    \begin{tikzpicture}
	\begin{axis}[
	    width=7cm,
	    height=4cm,
	    xmin= -1, xmax= 5,
	    ymin= 0, ymax = 2.,
	    axis lines = middle,
	    x label style={at={(axis description cs:1,-0.01)},anchor=north},
	    y label style={at={(axis description cs:0.15,1)},anchor=south},
	    xlabel={$x$},
	    ylabel={$U(x)$ },
	    xtick={ 0, \a+0.1, 1.9},
	    xticklabels={$0$, $a$, $b$},
	    ytick={0},
	    yticklabel={$0$},
	    ]
	    \addplot[domain=0.2:5, samples=500]{((x-\a)^2+0.2)*(x-\b)^2 + 0.3};
	    \addplot[densely dotted, samples=50, smooth,domain=0:6, name path=one] coordinates {(\a+0.1,0)(\a+0.1,1.)};
	    \addplot[densely dotted, samples=50, smooth,domain=0:6, name path=two] coordinates {(1.9,0)(1.9,1.5)};
	\end{axis}
    \end{tikzpicture}
    \caption{\scriptsize Potenziale al quale sono soggetti i camminatori.}
    \label{fig:lore}
\end{figure}

\noindent
Ipotizziamo di preparare il sistema nell'intervallo tra minimo e massimo del potenziale $[ a, b ]$, l'equazione dell'evoluzione del propagatore sarà:
\[
    \partial_{t}P = \partial_{x}\left(U'(x) P\right) + D\partial^2_{x^2}P
.\] 
Ed abbiamo visto che questa equazione ha come soluzione stazionaria:
\[
    P_s \approx N\exp\left[-\frac{U(x)}{D}\right]
.\] 
\tikzmath{\a = 1; \b = 3;}
\usetikzlibrary{patterns}
\usepgfplotslibrary{fillbetween}

\begin{figure}[H]
    \centering
    \begin{tikzpicture}
	\begin{axis}[
	    width=7cm,
	    height=4cm,
	    xmin= -1, xmax= 5,
	    ymin= 0, ymax = 1.8,
	    axis lines = middle,
	    x label style={at={(axis description cs:1,-0.01)},anchor=north},
	    y label style={at={(axis description cs:0.15,1)},anchor=south},
	    xlabel={$x$},
	    ylabel={$P_s(x)$ },
	    xtick={ 0, \a+0.1, 1.9},
	    xticklabels={$0$, $a$, $b$},
	    ytick={0},
	    yticklabel={$0$},
	    ]
	    \addplot[domain=0.2:5, samples=500]{e^(-((x-\a)^2+0.2)*(x-\b)^2 + 0.3)};
	    \addplot[densely dotted, samples=50, smooth,domain=0:6, name path=one] coordinates {(\a+0.1,0)(\a+0.1,0.65)};
	    \addplot[densely dotted, samples=50, smooth,domain=0:6, name path=two] coordinates {(1.9,0)(1.9,0.4)};
	\end{axis}
    \end{tikzpicture}
    \caption{\scriptsize Distribuzione di probabilità stazionaria.}
    \label{fig:lore}
\end{figure}

\noindent
A questo punto dobbiamo scegliere le condizioni al contorno su $a$ e $b$, prendiamo ad esempio le seguenti:
\begin{itemize}
    \item $b\equiv x_0$ bordo assorbente: le particelle che arrivano qui fanno "Puf".
    \item $a\equiv \to -\infty$ come dire bordo riflettente poiché a $-\infty$ c'è un muro di potenziale che va a $\infty$.
\end{itemize}
A questo punto possiamo prendere l'espressione \ref{eq:11_T_final} e specializzarla per il nostro problema. Si ottiene che il tempo di primo passaggio per andare da $a$ a $x_0$ vale:
\[\begin{aligned}
    &T(a\to x_0) =\\
    & = \lim_{a \to -\infty} \frac{1}{D}\int_{a}^{x_0} dy \exp\left(\frac{U(y)}{D}\right) \int_{a}^{y} \exp\left(-\frac{U(z)}{D}\right)dz 
.\end{aligned}\]
Mettiamoci nel limite in cui la barriera di potenziale è molto maggiore del coefficiente di diffusione:
\[
    \Delta  U = U(b) -U(a); \qquad \frac{\Delta U}{D}\gg 1
.\] 
Concentrandoci in un intorno di $b$ possiamo notare che:
\[
    \exp\left(\frac{U}{D}\right) \text{ ha max in }b
.\] 
Inoltre in questa approssimazione:
\[
    \text{Se } x = b \implies  \exp\left(-\frac{U}{D}\right)\to 0 \qquad \text{Con } \frac{U(b)}{D} \gg 1
.\] 
Visto che il secondo integrale contiene questo termine e che la variabile $y$ corre tra $-\infty$ e $x_0=b$ possiamo approssimare l'estremo di integrazione $y$ come:
\[
 \int_{-\infty}^{y} \exp\left(-\frac{U(z)}{D}\right)dz \sim \int_{-\infty}^{b} \exp\left(-\frac{U(z)}{D}\right)dz   
.\] 
Assumendo che i termini della somma provenienti da un intorno di $b$ contino poco. In questo modo i due integrali si disaccoppiano:
\[
    T \approx \frac{1}{D}\int_{-\infty}^{b}\exp\left(-\frac{U(z)}{D}\right)dz \int_{-\infty}^{x_0} dy \exp\left(\frac{U(y)}{D}\right)
.\] 
Un'altra approssimazione che si può fare è pensare $U(x)$ parabolico intorno ad $a$ e $b$:
\[\begin{aligned}
    &U(x)  \approx U(b) - \frac{1}{2} \frac{\left(x-b\right)^2}{\delta^2} \ \text{  vicino a } b\\
    &U(x)  \approx U(a) + \frac{1}{2} \frac{\left(x-a\right)^2}{\alpha^2} \ \text{  vicino ad } a
.\end{aligned}\]
Risolvendo quindi gli integrali arriviamo ad una forma per il tempo di primo passaggio:
\begin{greenbox}{Legge di Arrhenius}
\[
    T(a\to x_0) \approx 2\alpha\delta\pi  \exp\left(\frac{U(b) - U(a) }{D}\right)
.\] 
\'E una espressione simile alla legge di Arrhenius per le reazioni chimiche se poniamo $D = k_BT$.
\end{greenbox}
\noindent
\subsection{MFPT in più dimensioni}%
\label{sub:MFPT in più dimensioni}
Quando andiamo a studiare il caso multidimensionale si ha a che fare con questa equazione:
\begin{equation}
    \sum_{i}^{} A_i(x) \delta_iT(x)+\frac{1}{2}\sum_{i,J}^{} B_{iJ}\partial_{i}\partial_{J}T(x) = -1
    \label{eq:11_ mlti}
\end{equation}
Un modo elegante per risolvere è vederla come un problema agli autovalori. \\
Introduciamo il set di autofunzioni $Q_\lambda (x)$:
\[
    T(x) = \sum_{}^{} t_\lambda Q_\lambda(x) 
.\] 
Il problema si risolve reinserendo questa nella equazione \ref{eq:11_ mlti} e mettendo le opportune condizioni al contorno sulle $Q_\lambda$.\\
Procedendo in questo modo \ldots si può dimostrare che il tempo di primo passaggio prende la forma:
\[
    T(x) = \sum_{\lambda}^{} \frac{1}{\lambda}Q_\lambda (x) \int dx' P_\lambda (x')  
.\] 
Nei problemi tipici gli autovalori sono "separati esponenzialmente" l'uno dall'altro, quindi conta soltanto l'autovalore più basso (\ldots).\\
Il tempo di primo passaggio diventa quindi:
\[
    T(x) \approx \frac{1}{\lambda_1}Q_1\int P_1dx \approx \frac{1}{\lambda_1}
.\] 
\subsection{Calcolo numerico del MFTP}%
\label{sub:Calcolo numerico del MFTP}
Prendiamo la SDE per un set di camminatori:
\[
    dx = f(x) dt + g(x) d\omega
.\] 
Quindi per piccoli tempi possiamo scrivere che:
\[
    x_{n+1} = x_n + f(x_n) \Delta t + g(x_n) \Delta\omega
.\] 
Quindi mettiamoci in un punto $x_n$ e valutiamo la probabilità che $x_{n+1}$ sia fuori dal dominio considerato $(a,b)$. Ad esempio consideriamo la probabilità che $x_{n+1}$ sia oltre $b$.

Prendiamo il potenziale a doppia buca della sezione precedente, ipotizziamo che il camminatore elementare abbia fatto abbastanza passaggi da arrivare oltre il massimo $b$ e cadere nella seconda buca. \\
Possiamo chiederci quale sia la storia degli step effettuati da questo camminatore elementare, andando a vedere l'intensità del processo di Wiener in funzione della posizione si scopre che:
\begin{redbox}{}
Per superare il massimo del potenziale il processo di Wiener che spinge il camminatore deve essere sistematicamente diverso da zero.
\end{redbox}
\noindent
Ipotizziamo di avere l'andamento del processo stocastico per un camminatore $\omega (x) $, dimostriamo che tramite questo possiamo risalire al MFPT. 
\[
    x_{n+1} = x_n + f_n\Delta t + \sqrt{D}\Delta\omega
.\] 
E si ha anche che:
\[
    x_n = x_0 + \sum_{}^{} f_i \Delta t + \sqrt{D} \sum_{}^{} \Delta\omega_i
.\] 
Dalla prima possiamo estrarre $\Delta\omega_n$:
\[
    \Delta\omega_n = \frac{1}{\sqrt{D} }\left[x_{n+1}- \left(x_n + f_n \Delta t\right)\right]
.\] 
Sappiamo che la forma del processo di Wiener (la soluzione) è la seguente:
\[
P(\Delta\omega_n) \sim \exp\left(-\frac{\left(\Delta\omega_n\right)^2}{D\Delta t}\right)
.\] 
Per effettuare un salto da $a$ ad oltre il massimo $b$ abbiamo bisogno di una sequenza di salti giusti $\Delta\omega_i$, ovvero tali che:
\[
    x_0=a; \qquad x_n = b
.\] 
Quindi la probabilità di andare da $a$  a $b$ sarà la probabilità che tutti i processi di Wiener adeguati si verifichino:
\[
    P(a\to b) \sim \prod_{i}^{} \exp\left(-\frac{\left(\Delta\omega_i\right)^2}{D\Delta t}\right) 
.\] 
Ed inserendo la forma di $\Delta\omega_i$ ricavata in precedenza:
\[
    P(a\to b) \sim \prod_{i}^{} \exp\left(-\frac{\left( x_{i+1}-(x_i + f_i\Delta t) \right)^2}{D\Delta t}\right) 
.\] 
Passando al continuo nel tempo ed applicando dell'algebra se ne conclude che per trovare la probabilità massima di fare il passaggio basta risolvere:
\[
    \text{min}\int_{a}^{b} \left(\dot{x}-f\right) dt
.\] 
Questo determina la probabilità che si verifichi una speciale fluttuazione del processo di Wiener che ci fa fare il salto, in definitiva determina anche il tempo medio di primo passaggio.
\clearpage
